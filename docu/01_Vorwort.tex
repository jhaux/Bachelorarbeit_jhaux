
% Gutes Verständis, da Bodenphysik immer gute Beisiele hat:
% Salzsee, CO2 einlagerung
\label{cha:intro}

% Punshline
% \COT-Einlagerung ist ein aktuell sehr wichtiges Thema, dass noch nicht in allen Details verstanden ist. 

% Establishing shot: Überblick über Themen, die mit Transport durch Dichteinstabilitäten zu tun haben. GGf Klimawandel
Transportprozesse, die durch Dichteinstabilitäten hervorgerufen werde, also dadurch, dass ein schwereres Fluid auf einem leichteren aufschwimmt, sind Grund für eine ganze Reihe von Phänomenen.
Ein prominentes Beispiel für einen solchen Prozess kann man im Nordatlantik beobachten. Durch den Golfstrom wird warmes, salzhaltiges Wasser nach Norden transportiert, wo es durch Abkühlung und Verdunstung, dichter und salzhaltiger wird. Dies führt dazu, dass das Wasser schließlich absinkt. Dieser Prozess treibt die thermohaline Zirkulation an, eine Meeresströmung, die sich um die gesamte Erde spannt. 
Ähnliches passiert bei einem Salzsee, dessen Wasser langsam verdunstet. Da das Salz im Wasser nicht mit verdunstet, bildet sich eine Schicht schwereren Wassers an der Oberfläche, welche schließlich in tiefer Wasserschichten absinkt und ggf. auch in den Boden eindringt.

% Moving in: CO_2 Sequestration. Das Einlagern von CO2 Sie Kneafsy paper, bundestag
In Folge des Klimawandels und der Erkenntnis, dass der durch den Menschen verursachte \COT-Ausstoß diesen mit antreibt, wurden vermehrt Anstrengungen unternommen Wege zu finden, \COT langfristig einzulagern und somit der Atmosphäre zu entziehen. 
In einem Bericht aus dem Jahr 2008 des Ausschusses für Bildung, Forschung und Technikfolgenabschätzung, des deutschen Bundestags, wird festgestellt, dass ``noch erheblicher Bedarf an Forschung und Entwicklung besteht'' \citep{taccs}, bevor Verfahren zu Abtrennung, Transport und Einlagerung von \COT in tiefen Gesteinsschichten zur Verfügung stehen. Es werden noch ``15 bis 20 Jahre'' Arbeit dazu investiert werden müssen.

% Closeup: Lösungsverhalten von CO2, Wie schnell kann Wasser CO2 aufnehmen?
Eine nachhaltige Einlagerung von \COT kann beispielsweise in Form chemischer Bindung erfolgen. \cite{kneafsy} beschreiben in ihrer Arbeit aus dem Jahr 2010 einen Ansatz, bei dem \COT in tief gelegene, saline Aquifere geleitet wird. Dort bildet sich zunächst eine gasförmige Phase über dem Wasser aus, an deren Schnittstelle sich das \COT nach und nach im Wasser löst und schließlich in einer gesteinsförmigen Phase chemisch gebunden wird. 
% Der Prozess ist in Abbildung \ref{fig:sala} veranschaulicht.

Interessant für diese Arbeit ist der Prozess, wie das \COT im Wasser gelöst wird. Wasser, in welchem sich \COT gelöst hat, ist zwischen \SIrange{1}{10}{\percent} dichter als reines Wasser \citep{garcia}. Dadurch können Instabilitäten, wie oben beschreiben, entstehen, welche dazu führen, dass das \COT konvektiv, also schneller im Vergleich zu reiner Diffusion, von der Wasser-Luft Schnittstelle abgeführt wird.

% Was soll hier beobachtet werden?
Dieses Verhalten soll in der vorliegenden Arbeit untersucht werden, im Speziellen der Übergang von Diffusion zu Konvektion.
Dazu wird ein Experiment in einer \HSC durchgeführt, bei dem mit Hilfe eines Indikators angezeigt wird, wo sich im Verlauf der Zeit, die \COT-haltige Wasserphase befindet. 
Eine \HSC besteht aus zwei Glasplatten, welche sehr dicht aneinandergehalten werden. Dadurch ist die Fluid-Dynamik, welche sich dort abspielt, prinzipiell zweidimensional. Damit ist sowohl eine gute Beobachtung, als auch eine gute Beschreibung der auftretenden Phänomene möglich.
Für den Übergang zwischen diffusiver und konvektiver Phase wird erwartet, dass sich sogenannte Finger herausbilden, die in sich in regelmäßigen Abständen zueinander befinden, wie beispielsweise \cite{fernandez} beobachtet.

Neben den oben genannten Arbeiten gibt es noch einige weitere, wie man bei \cite{kneafsy} nachlesen kann, die sich mit dem Thema der \COT-Sequestration beschäftigen. Neben Experimenten wurden auch numerische Simulationen durchgeführt und mit den beobachteten Phänomenen verglichen. 

% Gliederung der Arbeit
% \newpage
Diese Arbeit ist in folgende Teile gegliedert: Zunächst soll in Teil \ref{cha:theo} eine theoretische Grundlage geschaffen werden und die gemachten Annahmen motiviert werden. Anschließend erklären die Teile \ref{cha:set} "`Experimenteller Aufbau"' und \ref{cha:meth} "`Bildanalyse"' die Methoden, mit denen Messdaten aufgenommen und ausgewertet wurden. Die Ergebisse dieser Messungen werden in Teil \ref{cha:res} "`Ergebnisse"' präsentiert und diskutiert. Am Ende folgt eine Zusammenfassunge mit Ausblick.

% \graph[./plot/saline_aquifer]{Chemische Bindung von \COT wird durch pumpen von \COT in tiefgelegene, saline Aquifere herbeigeführt. Dort löst sich das \COT im Wasser und wird später am Gestein gebunden.}{fig:sala}