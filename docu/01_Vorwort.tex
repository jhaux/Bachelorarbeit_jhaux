
% Gutes Verständis, da Bodenphysik immer gute Beisiele hat:
% Salzsee, CO2 einlagerung
\label{sec:intro}

Während der Durchführung meiner Bachelorarbeit entstanden zunächst Ideen zu einem Experiment, dass sich schließlich als nicht durchführbar herausstellte, für den zur Verfügung stehenden Zeitraum. In Folge dessen kam die Idee zu einem weiteren, für die zur Verfügung stehende Zeit besser geeignetem Experiment. Aus diesem Grund teiltsich diese Arbeit in jedem ihrer Abschnitte immer in inhaltlich dem einen, wie dem anderen Experiment zugehörigen Bereiche.

Grundlegend für alle Fragestellungen, die im Laufe dieser Arbeit aufkamen, sind Dichteinstabilitäten, die zur treibenden Kraft von Prozessen werden, die mit Hilfe einer Hele-Shaw-Zelle beobachtet werden sollen.
Zunächst wird die Frage gestellt, wie sich Verdunstungsphänomene auf Stofftransport in gesättigten, heterogenen, porösen Medien auswirken und zu Stofftransport von der Oberfläche in tiefere Schichten führt. Als Beispiel kann man sich einen Salzsee vorstellen, der dabei ist auszutrocknen und dabei Salz in tiefere Erdschichten einlagert.
In einem zweiten Ansatz wird die Frage gestellt, wie sich in Wasser lösendes \COT für Dichteinstabilitäten sorgt, die schließlich das gelöste Gas in tiefer Wasserbereiche führt. Auch hier lässt sich wieder ein sehr Anwendungsbezogenes Beispiel finden, wie schon \cite{fernandez} treffend festgestellt hat: Das Einlagern von \COT in Gesteinsschichten setzt vorraus, dass sich das \COT lange genug auf dem Gestein aufhält. Sorgt man dafür, dass unterirdische Wasserreservoirs mit \COT gesättigt werden kann man dieses Verhalten künstlich herbeiführen. Ein Verständnis dafür, wie sich \COT in Wasser löst und bewegt ist dafür grundlegend.

Diese Arbeit ist gegliedert in folgende Teile: Zunächst soll in Teil \ref{sec:theo} eine theoretische Grundlage geschaffen werden, zum Verständis der folgenden Abschnitte. Anschließend erkläre die Teile \ref{sec:set} "Experimenteller Aufbau" und \ref{sec:ima} "Bildanalyse" die Methoden, mit denen Messdaten beschaffen und ausgewerted wurden. Die Ergebisse dieser Messungen werden in Teil \ref{sec:res} "Ergebnisse" präsentiert und diskutiert. 
Am Ende folgt eine Zusammenfassunge mit Ausblick.