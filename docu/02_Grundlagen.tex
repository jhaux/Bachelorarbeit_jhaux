\label{cha:theo}

\TODO{neue Einleitung}
Zur Beschreibung der beiden durchgeführten \HSCs-Experimente sind zwei Themengebiete von Interesse: Zum einen wird ein Ansatz benötigt um poröse Medien zu charakterisieren, zum anderen muss verstanden werden, wie das Ausbilden von Dichteinstabilitäten von statten geht und wie sich beispielsweise dadurch Finger bilden.


\section{Fluiddynamik}
\label{sec:hyd}
Die Bewegung inkompressibler Fluidpartikel wird durch die \NSG beschrieben:
\begin{eqnarray}
 \rho \dot{\vec{v}} = \rho \left( \frac{\partial \vec{v}}{\partial t} + \vec{v} \cdot \nabla \vec{v} \right) = - \nabla p + \mu \nabla^2 \vec{v} + \vec{f}
\end{eqnarray}
Hierbei bezeichnet $\vec{v}$ die Geschwindigkeit des betrachteten Partikels, $\nabla p$ den Druckgradienten und $\mu \nabla^2 \vec{v}$ den Reibungsterm. $\vec{f}$ steht für äußere Kräfte, die auf das System wirken, wie zum Beispiel die Gravitations- und Corioliskraft. \citep{roth2005}

Eine bessere Einschätzung der Einflüsse der einzelnen Terme wird über die dimensionslose Formulierung der \NSG möglich. Dazu werden die Größen $t' = \frac{t}{\tau}$,  $\vec{x}' = \frac{\vec{x}}{l}$ und $\vec{v}' = \frac{\vec{v}}{u}$, mit chrakteristischer Zeitspanne $\tau$, Länge $l$ und Geschwindigkeit $u$, definiert, mit den zugehörigen Differntialoperatoren $\partial_{t'} = \tau\partial_t$ und $\nabla' = l\nabla$. 
Des weiteren wird für die äußeren Kräfte die Gravitationskraft eingesetzt: $\vec{f} = \rho\vec{g}$

Damit erhält man
\begin{equation}
\begin{aligned}
 \mathrm{St} \, \partial_{t'} \vec{v}' &+ \left[\vec{v}' \cdot \nabla' \right]\vec{v}' \\
 &= -\nabla'p' + \frac{1}{Re}\nabla'^2\vec{v}' + \frac{1}{\mathrm{Fr}}\hat{g}
\end{aligned}
\end{equation}

mit $\hat{g} = \frac{\vec{g}}{g}$, $p' = \frac{p}{u^2\rho}$ und den dimensionslosen Größen 
\begin{align}
 \mathrm{St} := \frac{l}{u\tau}, \; \mathrm{Fr} := \frac{u^2}{lg} \; \mathrm{und}  \; \mathrm{Re} := \frac{\rho u l}{\mu}
 \label{eq:Re}
\end{align}
 mit den Namen \textit{Strouhal}-, \textit{Froude}-, und \textit{Reynoldszahl}.

Eine Abschätzung der auftretenden Geschwindigkeiten und damit der Reynoldszahl Re in Teil \ref{cha:set} zeigen, dass das durchgeführte Experiment im Stokes Regime stattfindet. Der Trägheitsterm $\left[ \vec{v} \cdot \nabla \right] \vec{v}$  ist vernachlässigbar \citep{roth2005}.




% \TODO{brauche ich das???}
% \TODO{
% Kann sich die Luft in den wasserfreien Poren frei bewegen und ist ihr Druckgradient vernachlässigbar klein gegenüber dem des Wasser, lässt sich der Fluss 
% mit dem \textit{Buckingham-Darcy Gesetz} beschreiben:
% \begin{align}
%  \vec{j}_w = -K\nabla\Phi = -K \left[ \nabla \Phi_m - \rho \vec{g} \right]
% \end{align}
% Mit $\Phi$ wird das Wasserpotential bezeichnet, mit $\Phi_m$ das Matrixpotential, welches dem Druckunterschied am Wasser-Luft Übergang entspricht.
% }


\section{Poröse Medien}
\label{sec:por}

% \graph[./plot/porous]{Schematische Darstellung eines porösen Mediums. Grau dargestellt ist die Matrix (1), der feste Bereich des Mediums. Dazwischen befinden sich die zunächst leeren Poren (2).}{fig:por}
Poröse Medien, wie beispielsweise Sand oder Ton, teilen den Raum, den sie füllen, in Matrix und Poren auf \citep{roth2005}.  Matrix bezeichnet in diesem Fall die Bereiche, an dem sich das feste Medium befinden, der eigentliche Sand, beispielsweise. Die Poren bilden den zunächst leeren Raum dazwischen.
Als charakteristische Größen lassen sich die Porosität $\phi$ und die Raumdichte $\rho_{bulk}$ definieren:
\begin{align}
 \phi &= \frac{V_{pore}}{V_{tot}} \\
 \rho_{bulk} &= \frac{m_{matrix}}{V_{tot}} % = \frac{\phi}{1-\phi} % Keiner weiß was das sein soll!
\end{align}
$V_{pore}$ und $V_{tot}$ bezeichnen die Volumina, die von den Poren, beziehungsweise insgesamt eingenommen werden, $m_{matrix}$ die Masse der Matrix. 

Möchte man die Bewegung der in Abschnitt \ref{sec:hyd} genannten Fluidpartikel in porösen Medien beschreiben, wird zusätzlich die Information benötigt, dass sich die Partikel nur durch die Poren bewegen können:
\mbox{$\theta = \frac{V_{water}}{V_{tot}}$} bezeichnet den Wassergehalt, $\Theta = \frac{\theta}{\phi}$ die Sättigung des betrachteten Mediums. 
Dabei gilt $\theta \leq \phi$. Beträgt die Sättigung weniger als 1 ist nicht der gesamte Porenraum mit Wasser gefüllt. Für $\theta = \phi$ reicht es nur die
Bewegung der Flüssigkeit um die Matrix zu beschreiben. Der Fluss des Wassers durch die Poren wird durch $\vec{j}_w = \theta \cdot \vec{v}$ beschrieben. 
% Aus der Massenerhaltung folgt:
% \begin{align}
%  \frac{\partial \theta}{\partial t} = \nabla \vec{j}_w
% \end{align}
Mit der hydraulischen Leitfähigkeit $K$ stellt \textit{Darcy's Gesetz} eine Verbindung zwischen Fluss und Einwirkungskräften her:
\begin{align}
 \vec{j}_w = -K \left[ \Delta p - \rho \vec{g} \right]
 \label{darcy}
\end{align}


\section{Transport von gelösten Stoffen}
\label{sec:soltra}

Der Transport von gelösten Stoffen in porösen Medien wird dominiert durch die Bewegung des Wassers. Allerdings spielt auch Diffusion eine Rolle. Gelöste Stoffe können Tracer wie \BCG sein, aber auch Gase wie \COTn.
Zwei Prozesse dominieren also das durchgeführte Experiment zu verschiedenen Zeiten: Konvektion und Diffusion. Eine Abschätzung, wann welcher Prozess überwiegt kann mit Hilfe der Rayleighzahl erfolgen.

\subsection{Konvektion}
\label{sec:conv}
Mit Konvektion wird der Transport von Masse durch Bewegung eines Fluids bezeichnet. Dominiert nur dieser Effekt, bewegen sich alle Tracerpartikel entlang der Strömungsrichtung des Wassers.
Das Geschwindigkeitsfeld des Wassers wird durch das Vektorfeld $\vec{v}(\vec{x})$ beschrieben, der Verteilung des Tracers durch das Skalarfeld $C(\vec{x})$. 
Die Änderung der Konzentration des Tracers durch die Bewegung des Fluids wird somit durch folgende Differentialgleichung beschrieben:
\begin{align}
 \frac{\partial C}{\partial t} = -\vec{v} \, \vec{\nabla}C
\end{align}


% \subsection{Dispersion}
% \label{sec:disp}
% Mit Dispersion bezeichnet man in unserem Fall den Effekt, dass Tracerpartikel im Porenraum trotz nächster Nähe zueinander unterschiedlich schnell sein können. Dieser Effekt wird auch Tayler-Aris Dispersion bezeichnet und kommt dadurch zustande, dass sich das Fluid aufgrund von Reibungseffekten am Rand der Poren langsamer bewegt als in deren Mitte.

\subsection{Diffusion}
\label{sec:diff}
Diffusion, also die Bewegung eines Partikels entlang des Konzentrationsgradienten, wird durch die Brownsche Molekularbewegung verursacht. Durch Diffusion wird also der Konzentrationsgradient, auch ohne vorhandene Bewegung des Fluids, ausgeglichen. Der Partikelfluss $\vec{j}_C$ wird durch das erste Ficksche Gesetz beschrieben:
\begin{align}
 \vec{j}_C = D_m \vec{\nabla} C
\end{align}
Ebenfalls mit Hilfe des molekularen Diffusionskoeffizienten $D_m$ beschreibt das zweite Ficksche Gesetz die Konzentrationsänderung über die Zeit:
\begin{align}
 \frac{\partial C}{\partial t} = D_m \cdot \vec{\nabla}^2 C
\end{align}

Das poröse Medium schränkt den Raum, in den die Tracerpartikel diffundieren können ein. Millington und Quirk \citeyearpar{milli-quir} definieren hierfür zwei 
effektive Diffusionskoeffizienten:
\begin{align}
 D_{eff, 1}^{diff} &= D_m \cdot \frac{\theta^{\frac{7}{3}}}{\phi^2} \\
 D_{eff, 2}^{diff} &= D_m \cdot \frac{\theta}{              \phi^\frac{3}{2}}
\end{align}
Da in allen Experimenten Sättigung vorliegt \mbox{($\Theta = 1$)} kann die obige Gleichung umgeschrieben werden zu
\begin{align}
 D_{eff}^{diff} &= D_m \cdot \phi^{\frac{1}{3}}
 \label{eq:Deff}
\end{align}

\subsection{Rayleigh Zahl}
\label{sec:ray}

Zur Abschätzung, wann in den \HSCsm Experimenten konvektive gegenüber diffusiven Prozessen überwiegen und umgekehrt, eignet sich die Rayleighzahl $\mathrm{Ra} = \frac{U \cdot e}{D}$. Hierbei stellt $U \cdot e$ eine Abschätzung des Anteils konvektiver Prozesse, in Form einer Auftriebsgeschwindigkeit $U$ multipliziert mit der Spaltbreite $e$ dar, und $D$ in Form des Diffusionskoeffizienten, eine Abschätzung des Anteils an diffusiven Prozessen. Die Spaltbreite ist die charkteristische Länge in der \HSCn.
$D$ beläuft sich im Falle von gelöstem \COT auf $D = \SI[round-precision=2]{1,97e-9}{\meter\squared\per\second}$ \citep{frank}.
\cite{fernandez} geben in Ihrer Arbeit als Abschätzung für $U$ in einer \HSC folgendes an:
\begin{align}
 U = \frac{\Delta\rho g K}{\mu}
 \label{eq:U}
\end{align}

Als Parameter gehen der Dichteunterschied $\Delta\rho$ zwischen reinem Wasser und solchem, in dem sich \COT gelöst hat, die Erdbeschleunigung $g$, die Spaltbreite $e$ sowie die Viskosität $\mu$ und die hydraulische Konduktivität $K = \frac{e^2}{12}$, ein.

Damit erhalten sie die dimensionslose Rayleighzahl:
\begin{align}
 \mathrm{Ra} = \frac{\Delta\rho g e^3}{12 \mu D}
 \label{eq:Ra1}
\end{align}

Diese Abschätzung gilt für den Fall, dass die charakteristische Länge des Systems die Spaltbreite ist. Ist die Zelle jedoch mit einem porösen Medium gefüllt, gilt dies nicht mehr.
Eine sinnvolle Abschätzung von Ra kann dann mit Hilfe der Leitfähigkeit $K = \frac{r_{pore}^2}{8\mu}$ erfolgen:
\begin{align}
 \mathrm{Ra}_{pore} = \frac{r_{pore}^2 g h \Delta \rho}{8 \mu D}
 \label{eq:Ra2}
\end{align}

Die Höhe der Schicht, die durch Diffusion von \COT im Wasser entsteht, geht mit dem Parameter $h$ in die Gleichung ein.
$K = \frac{r_{pore}^2}{8\mu}$ leitet sich aus dem Gesetz von Hagen-Poiseuille her, angewandt auf eine Röhre mit Radius $r_{pore}$. Damit gilt $j = -\frac{r_{pore}^2}{8\mu} \nabla p$. Zusammen mit Darcy's Gesetz erhält man daraus $K$. Der Porenradius $r_{pore}$ kann unter der Annahme, dass die Kugeln perfekt aufeinander geschichtet sind, abgeschätzt werden. In diesem Fall gilt $\phi=0,35$ \citep{song}, woraus man, über den Kugelradius $\bar{r}$, $r_{pore} \approx 0,11 \cdot \bar{r}$ erhält.


Laut \cite{kneafsy} überwiegen konvektive Prozesse ab einer Rayleighzahl von \mbox{$\mathrm{Ra} > 4\pi^2$}.
