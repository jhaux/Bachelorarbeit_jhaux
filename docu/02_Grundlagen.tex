\label{cha:theo}

Zur Beschreibung der beiden durchgeführten \HSCs-Experimente sind zwei Themengebiete von Interesse: Zum einen wird ein Ansatz benötigt um poröse Medien zu
charakterisieren, zum anderen muss verstanden werden, wie das Ausbilden von Dichteinstabilitäten von statten geht und wie sich beispielsweise dadurch Finger
bilden.


\section{Poröse Medien}
\label{sec:por}

\graph[./plot/porous]{Schematische Darstellung eines porösen Mediums. Grau dargestellt ist die Matrix (1), der feste Bereich des Mediums. Dazwischen befinden sich die zunächst leeren Poren (2).}{fig:por}
Poröse Medien, wie beispielsweise Sand oder Ton, teilen den Raum, den sie füllen, in Matrix und Poren auf \citep{roth2005}.  Matrix bezeichnet in diesem Fall
die Bereiche, an dem sich das feste Medium befinden, der eigentliche Sand. Die Poren bilden den zunächst leeren Raum dazwischen.
Als charakteristische Größen lassen sich die Porosität $\phi$ und die Raumdichte $\rho_{bulk}$ definieren:
\begin{align}
 \phi &= \frac{V_{pore}}{V_{tot}} \\
 \rho_{bulk} &= \frac{m_{matrix}}{V_{tot}} = \frac{\phi}{1-\phi}
\end{align}
$V_{pore}$ und $V_{tot}$ bezeichnen die Volumina, die von den Poren, bzw insgesamt eingenommen werden, $m_{matrix}$ die Masse der Matrix. 

\TODO{Abschätzung Porenradius (Lisa S. 49)}
\section{Hydraulik}
\label{sec:hyd}
Die Bewegung inkompressibler Fluidpartikel wird durch die Navier-Stoke Gleichung beschrieben:
\begin{eqnarray}
 \rho \dot{\vec{v}} = \rho \left( \frac{\partial \vec{v}}{\partial t} + \vec{v} \cdot \nabla \vec{v} \right) = - \nabla p + \mu \nabla^2 \vec{v} + \vec{f}
\end{eqnarray}
Hierbei bezeichnet $\vec{v}$ die Geschwindigkeit des betrachteten Partikels, $\nabla p$ den Druckgradienten und $\mu \nabla^2 \vec{v}$ den Reibungsterm. $\vec{f}$ steht für äußere Kräfte, die auf das System wirken, wie zum Beispiel die Gravitationskraft. \citep{roth2005}

Möchte man die Bewegung oben genannter Fluidpartikel in porösen Medien beschreiben werden zusätzliche Informationen benötigt, das sich die Partikel nur durch die Poren bewegen können:
$\theta = \frac{V_{water}}{V_{tot}}$ bezeichnet den Wassergehalt, $\Theta = \frac{\theta}{\phi}$ die Sättigung des betrachteten Mediums. 
Dabei gilt $\theta \leq \phi$. Beträgt die Sättung weniger als 1 ist nicht der gesamte Porenraum mit Wasser gefüllt. Für $\theta = \phi$ reicht es nur die
Bewegung der Flüssigkeit um die Matrix zu beschreiben. Der Fluss des Wassers durch die Poren wird durch $\vec{j}_w = \theta \cdot \vec{v}$ beschrieben. 
Aus der Massenerhaltung folgt:
\begin{align}
 \frac{\partial \theta}{\partial t} = \nabla \vec{j}_w
\end{align}
Mit der Leitfähigkeit $K$ stellt \textit{Darcy's Gesetz} eine Verbindung zwischen Fluss und Einwirkungskräften her:
\begin{align}
 \vec{j}_w = -K \left[ \Delta p - \rho \vec{g} \right]
\end{align}

% \TODO{brauche ich das???}
% \TODO{
% Kann sich die Luft in den wasserfreien Poren frei bewegen und ist ihr Druckgradient vernachlässigbar klein gegenüber dem des Wasser, lässt sich der Fluss 
% mit dem \textit{Buckingham-Darcy Gesetz} beschreiben:
% \begin{align}
%  \vec{j}_w = -K\nabla\Phi = -K \left[ \nabla \Phi_m - \rho \vec{g} \right]
% \end{align}
% Mit $\Phi$ wird das Wasserpotential bezeichnet, mit $\Phi_m$ das Matrixpotential, welches dem Druckunterschied am Wasser-Luft Übergang entspricht.
% }


\section{Transport von gelösten Stoffen}
\label{sec:soltra}

Der Transport von gelösten Stoffen in porösen Medien wird dominiert durch die Bewegung des Wassers. Allerdings spielen auch Diffusion und Dispersion eine Rolle.
Gelöste Stoffe können Tracer wie \BB oder \BCG sein, aber auch Gase wie \COT.

\subsection{Konvektion}
\label{sec:conv}
Mit Konvektion wird der Transport von Masse durch Bewegung eines Fluids bezeichnet. Dominiert nur dieser Effekt, bewegen sich alle Tracerpartikel entlang der Strömungsrichtung des Wassers.
Das Geschwindigkeitsfeld des Wassers wird durch das Vektorfeld $\vec{v}(\vec{x})$ beschrieben, der Verteilung des Tracers durch das Skalarfeld $C(\vec{x})$. 
Die Änderung der Konzentration des Tracers durch die Bewegung des Fluids ist somit durch folgende Differnetialgleichung:
\begin{align}
 \frac{\partial C}{\partial t} = -\vec{v}\vec{\nabla}C
\end{align}


\subsection{Dispersion}
\label{sec:disp}
Mit Dispersion bezeichnet man in unserem Fall den Effekt, dass Tracerpartikel im Porenraum trotz nächster Nähe zueinander unterschiedlich schnell sein können. Dieser Effekt wird auch Tayler-Aris Dispersion bezeichnet und kommt dadurch zustande, dass sich das Fluid aufgrund von Reibungseffekten am Rand der Poren langsamer bewegt als in deren Mitte.

\subsection{Diffusion}
\label{sec:diff}
Diffusion, also die Bewegung eines Partikels entlang des Konzentrationsgradienten, wird durch die Brownsche Molekularbewegung verursacht. Durch Diffusion wird
der Konzentrationsgradient, auch ohne vorhandene Bewegung des Fluids, stattfinden. Der Partikelfluss $\vec{j}_C$ wird durch das erste Ficksche Gesetz
beschrieben:
\begin{align}
 \vec{j}_C = D \vec{\nabla} C
\end{align}
Mit Hilfe des molekularen Diffusionskoeffizienten beschreibt das zweite Ficksche Gesetz die Konzentrationsänderung über die Zeit:
\begin{align}
 \frac{\partial C}{\partial t} = D_m \cdot \vec{\nabla}^2 C
\end{align}

Das poröse Medium schränkt den Raum, in den die Tracerpartikel diffundieren können ein. Millington und Quirk \citeyearpar{milli-quir} definieren hierfür zwei 
effektive Diffusionskoeffizienten:
\begin{align}
 D_{eff}^{diff} &= D_m \cdot \frac{\theta^{\frac{7}{3}}}{\phi^2} \\
 D_{eff}^{diff} &= D_m \cdot \frac{\theta}{              \phi^\frac{3}{2}}
\end{align}
Da in allen Experimenten Sättigung vorliegt ($\Theta = 1$) kann die obige Gleichung umgeschrieben werden zu
\begin{align}
 D_{eff}^{diff} &= D_m \cdot \phi^{\frac{1}{3}}
\end{align}

\subsection{Rayleigh Zahl}
\label{sec:ray}
Zur Abschätzung, wann in den \HSC-Experimenten konvektive gegenüber diffusiven Prozessen und umgekehrt überwiegen, schlagen \citeauthor*{fernandez}

\TODO{solute transport}
\TODO{Konduktivität und Diffusivität => Reynolds}
