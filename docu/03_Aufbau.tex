\label{cha:set}
\widegraph[./plot/Aufb_allg_num]{Grundsätzlicher Aufbau der beiden durchgeführten Experimente. Zu sehen sind: 1: \HSC, 2: Lichtquelle, 3: Kasten, 4: Lüfter, 5: Computer, 6: \COTm Behälter, 7: Pumpe, 8: Kamera, 9: 3-Wege-Ventil, 10: Reservoir zum Halten des Wasserspiegels bei langen Experimenten.}{fig:auf}
% \graph[./plot/Aufb_allg_num]{Grundsätzlicher Aufbau der beiden durchgeführten Experimente. Zu sehen sind: 1: \HSC, 2: Lichtquelle, 3: Kasten, 4: Lüfter, 5: Computer, 6: \COTm Behälter, 7: Pumpe, 8: Kamera, 9: 3-Wege-Ventil, 10: Reservoir zum Halten des Wasserspiegels bei langen Experimenten.}{fig:auf} % nutzen für onecolumn


Für beide in Teil \ref{cha:intro} beschriebenen Fragestellungen ist die Dynamik der betrachteten Systeme interessant. Daher wird jeweils ein ``\LTM''
durchgeführt. Hierzu wird eine \HSC vor einer homogenen Lichtquelle platziert. Das Licht, das die Zelle durchdringt, wird von einer Digitalkamera aufgezeichnet
und für die spätere Auswertung gespeichert.
\TODO{Beschreiben was ich beobachten will}

\subsection{\HSC}
\label{sec:hsc}
Grundsätzlich besteht die Zelle aus zwei Glasplatten, die einem kleinen Abstand zueinander parallel angeordnet sind. Bei diesem Aufbau ist der Zwischenraum an
drei der vier Seiten abgedichtet, sodass kein Wasser abfließen kann. Dies wird gewährleistet, indem die Glasplatten mit Hilfe von Keilen in einen Rahmen und
gegen die Dichtungsgummis, welche auch als Abstandhalter dienen, gepresst werden. 
Die offene Seite der Zelle zeigt nach oben. Am unteren Ende der Zelle befindet sich ein Ausfluss, über den die Zelle kontrolliert mit Wasser oder einer
gewünschten Lösung befüllt werden kann.
Die Abmessungen der verwendeten Zellen sind in Tabelle \ref{tab:Hdim} festgehalten.

% \mediumgraph[./plot/happy_cell_dimensions]{Abmessungen der \HSC. Ansicht von oben und von der Seite. 1: Keil, 2: Dichtung und Abstandhalter, 3: Rahmen, 4: Glasplatte, 5: Füllung der Zelle. Man bemerke, dass die Zelle glücklich ist!}{fig:Hdim}
\mediumgraph[./plot/cell_dimensions]{Abmessungen der \HSC. Ansicht von oben und von der Seite. 1: Keil, 2: Dichtung und Abstandhalter, 3: Rahmen, 4: Glasplatte, 5: Füllung der Zelle.}{fig:Hdim}

Der Vorteil einer \HSC ist, dass man mit ihr Beobachtungen zweidimensionaler Natur machen kann.


\begin{table}[b]
  \begin{tabularx}{\linewidth}{X|c|c|c} %>{\raggedright}p{2,2cm}<{}
    Aufbau			& Höhe				& Breite			& Spaltbreite \\
    \hline\hline
%     Verdunstungs\-experi\-ment	& \SI{ 500}{\milli\meter}	& \SI{273}{\milli\meter}	& \SI{3}{\milli\meter} \\
%     \hline
    \COTm 			& \SI{ 250}{\milli\meter}	& \SI{273}{\milli\meter}	& \SI[round-precision=1]{2,1}{\milli\meter} \\
    Experiment			& \SI{ 500}{\milli\meter}	& \SI{273}{\milli\meter}	& \SI[round-precision=1]{2,1}{\milli\meter}
  \end{tabularx}
  \caption{Dimensionierung der \HSCs für die beiden durchgeführten Experimente. Siehe auch Abbildung \ref{fig:Hdim}.}
  \label{tab:Hdim}
\end{table}

\subsection{Kamera}
\label{sec:cam}
Die Messung wird mit Hilfe einer \TODO{AVT Pike F-505B}-Kamera durchgeführt. Diese wurde schon von \cite{heberle} zum Einsatz gebracht und ausführlich
beschrieben. 

Die Daten in Tabelle \ref{tab:cam} sind aus dieser Arbeit sowie dem Internetauftritt des Herstellers \citep{pike_sheet} entnommen. 
\begin{table}[b]
 \begin{tabularx}{\linewidth}{X|X}
  Komponente	& Eigenschaft \\
  \hline\hline
  Kamera	& AVT Pike F-505B \\
  Sensortyp	& CCD \\
  Farbtiefe	& \SI{14}{bit}, monochrom \\
  Auflösung	& 2452 $\times$ 2054 Pixel \\
  Schnittstelle	& IEEE1394-B \\
  \hline
  Objektiv	& Fujinon HF50SA-1 \\
  Brennweite	& 50 mm \\
  \hline
%   \TODO{Filter 1}	& \SI{452}{\nano\meter}, FWHM \SI{9}{\nano\meter} \\
  Filter 2	& \SI{632}{\nano\meter}, FWHM: \SI{11}{\nano\meter} 
 \end{tabularx}
 \caption{Herstellerangaben zur verwendeten Kamera, sowie des Objektives und der Filter.}
 \label{tab:cam}
\end{table}
Die Kamera kann über eine Firewire Schnittstelle gesteuert und ausgelesen werden. Auch das Auswählen des benötigten Filters kann mit Hilfe eines Filterrads
vom Computer aus geschehen.


\subsection{Lichtquelle}
\label{sec:light}
Zur Durchleuchtung der Zelle wird ein Array aus LEDs der Farben Rot, Grün und Blau verwendet. Eine davor gespannte Diffusorfolie sorgt für räumlich gleichmäßige Beleuchtung. Die Lichtquelle befindet sich zusätzlich in einem mit Aluminiumfolie ausgekleideten Kasten, in welchen auch Lüfter eingebaut sind. Per Computer lassen sich die LEDs zusammen mit der Lüftung ein und ausschalten. Auch hierzu finden sich wieder ausführliche Informationen bei \cite{buchner} und \cite{heberle}.


% \section{Verdunstungsexperiment}
% \label{sec:eva}
% 
% % \graph{Aufbau des Verdunstungsexperiments. Zu Sehen sind 1. , 2. , ...}{gr:suevp}
% % \TODO{Was will ich beobachten?}
% Verdunstungsexperimente mit einer \HSC haben das Problem, dass der Austausch zwischen Zelle und umgebender Luft durch den Plattenabstand der Zelle eingeschränkt wird. Dieses Experiment soll eine Abschätzung dazu liefern, wie die Zeitskalen für eine \HSC-Experiment mit porösem Medium sind.
% 
% Für diesen Versuch wird ein bereits vorhandener Aufbau von \cite{feustel} verwendet, der aus einem porösen Medium mit räumlich stark heterogenen 
% Eigenschaften besteht. Hierbei ist die große \HSC ist mit Glaskügelchen verschiedener Größen gefüllt. Dabei enstehen Regionen, die sich jeweils über einen Teil 
% der Zellbreite erstrecken, wie in Abbildung \ref{fig:het} zu sehen ist. 
% Die Größen der verwendeten Kugeln (\textit{SiLi-Beads}) der Firma \textit{Sigmund-Lindner GmbH} sind in Tabelle \ref{tab:kug} notiert. 
% 
% \smallgraph[./plot/hetero]{Die heterogene Struktur des aufgeschütteten porösen Mediums. Die verschiedenen Bereiche unterschiedlicher Kugelgröße lassen sich mit Hilfe von Tabelle \ref{tab:kug} zuordnen. Die Ausbreitung des Tracers ist mit einer gestrichelten weißen Linie markiert. Diese Aufnahme wurde mit einem vorgeschalteten \SI{630}{\nano\meter}-Filter gemacht, wodurch die Ausbreitung besser hervorgehoben wird.}{fig:het}
% 
% \begin{table}[b]
%   \begin{tabularx}{\linewidth}{X|c|c|c}
% 		& Durch\-messer 			& Stdabw. d. \O{}			& Raumgewicht	\\
% 		& $\left[\si{\milli\meter}\right]$	& $\left[\si{\milli\meter}\right]$	& $\left[\si{\kg\per\dm\tothe{3}}\right]$ \\
%     \hline\hline
%     \circled{1}	& 0,07 - 0,11				& 0,06					& 1,37 \\
%     \circled{2}	& 0,2 - 0,3				& 0,03					& 1,44 \\
%     \circled{3}	& 0,4 - 0,6				& 0,21					& 1,47 
%   \end{tabularx}
%   \caption{Daten der verwendeten \textit{SiLi-Beads}. Die Materialdichte der Kugeln betragt \SI{2,5}{\kg\per\dm\tothe{3}}. Entnommen aus \cite{feustel}.}
%   \label{tab:kug}
% \end{table}
% 
% Für dieses Experiment wurde der Zufluss am unteren Rand der Zelle so eingerichtet, dass sowohl \COT als auch Wasser oder gelöstes \BB kontrolliert in die Zelle geleitet werden können. \BB ist eine Nahrungsmittelfarbe, welche hier in einer Konzentration von \SI{0,05}{\gram\per\liter} verwendet wird. Der Farbstoff absorbiert im Wellenlängenbereich von \SI{630}{\nano\meter} maximal. Eine \BB Lösung kann also gut mit dem passenden Filter vor der Kamera verfolgt werden. Siehe auch hierzu Abbildung \ref{fig:het}. Um Lufteinschlüsse zwischen den Glaskugeln beim Befüllen der Zelle zu vermeiden wird vor dem Fluten mit Wasser \COT durch die Zelle gespült. Dazu wird eine Flasche mit etwas Trockeneis befüllt und anschließend das entstehende Gas in die Zelle geleitet. Dazu wird die Flasche, mit der Öffnung leicht nach unten geneigt, auf den Kasten, in dem sich die Zelle befindet gelegt. Ein Schlauch verbindet Flaschenöffnung und untere Zufluss, sodass das Gas einfach durch die Zelle hindurchfließt, da gasförmiges \COT schwerer als Luft ist. Nach etwa ein bis zwei Stunden wird angenommen, dass die gesamte Luft aus der Zelle verdrängt wurde und es wird Wasser zugeführt. In diesem löst sich das Gas, sodass das poröse Medium komplett mit einer Wasser-\COTm Lösung gefüllt ist. Um unerwartete Effekte, ggf. ausgelöst durch das \COT, zu vermeiden wird anschließend noch mehrmals Wasser durch die Zelle gepumpt, damit möglichst kein gelöstes \COT in der Zelle zurückbleibt. Zum Beginn des Experiments wird der Wasserspiegel auf Höhe der obersten Kugelschicht gesenkt und die Pumpschläuche entfernt.
% Die Kamera filmt das Experiment mit einer Bildrate von \SI{1}{Bild\per\hour}. Ein Skript steuert die Beleuchtung, sodass diese nur während der Bildaufnahme angeschaltet ist.



\section{\COTm Experiment}
\label{sec:cot}
Für das zweite Experiment wurde die \HSCs mit einer \BCG Lösung mit einer Konzentration von \SI{3,5}{\milli\gram\per\liter} befüllt. 
\BCG ist eine Indikatorlösung, die von blau zu gelb umschlägt, wenn sich der pH-Wert in einem Bereich von 5,4 bis 3,9 ändert. Reines Wasser, das nur mit Luft in Kontakt ist, hat einen pH-Wert von 5,6, wohingegen Wasser, in dem sich \COT gelöst hat, einen pH-Wert von 3,9 annimmt. Die ursprünglich fast neutrale Lösung absorbiert stark im \SI{630}{\nano\meter}-Bereich während nach dem pH-Wert induzierten Farbumschlag das Maximum der Absorption bei \SI{450}{\nano\meter} liegt. Dieses Verhalten wird in Abbildung \ref{fig:bcg} deutlich gemacht.

\graph[./plot/BCG]{Absorbierende Eigenschaften von \BCG. \citep{bcg:wiki}}{fig:bcg}

Diese Eigenschaft passt sehr gut zu den zur Verfügung stehenden Filtern und macht es möglich zu verfolgen wo \COT in Wasser gelöst ist.
Die Kamera filmt wieder das Experiment, dieses mal jedoch mit einer sehr viel höheren Bildrate von ca. \SI{1}{Bild\per\minute}.  Es werden Aufnahmen mit dem \SI{630}{\nano\meter}- und dem \SI{450}{\nano\meter}-Filter gemacht, sowie Aufnahmen bei verschlossenem Objektiv, die später zur Dunkelstromkorrektur verwendet 
werden sollen. 
Nachdem die Kamera die ersten Bilder geschossen hat, wird von oben \COT in die Zelle geleitet. Auch hier geschieht dies mit Hilfe von 
Trockeneis, dieses Mal allerdings wird das freigesetzte Gas bei niedriger Rate gepumpt. Damit sich in dem Behältnis für das Gas wirklich nur \COT befindet, ist dieses nach oben hin geöffnet, sodass die leichtere Luft verdrängt wird. Zwischen Lösung und oberer Kante der Glasplatten wurde ausreichend Platz gelassen, sodass sich eine breitere \COTm Schicht bilden kann.

Mit Gleichung \ref{eq:U} und \ref{eq:Re} kann abgeschätzt werden, ob das Experiment im Stokes Regime durchgeführt wird. Als Parameter werden 
\begin{align*}
  \rho &= \rho_{Wasser} = \SI{1}{\g\per\cubic\centi\meter}, \\ 
  \Delta\rho &= \left| \rho_{Wasser} - \rho_{\mathrm{CO}_{2, aq}} \right| = \SI[round-precision=4]{0,0017}{\gram\per\cubic\centi\meter} \\
  g &= \SI[round-precision=2]{9,81}{\meter\per\squared\second} \\
  e &= \SI[round-precision=1]{2,1}{\milli\meter} \\
  \mu &= \mu_{Wasser} = \SI{8,9e-4}{\pascal\second}
\end{align*}

gewählt. Damit erhält man
\begin{equation}
 \mathrm{Re} = 0,096 < 1
\end{equation}
was sehr viel kleiner ist als die kritische Reynoldszahl $Re_{crit} = ???$, ab der das System turbulent werden kann. Es liegt bei dem Experiment, wie es hier durchgeführt wird Stokes-Fluss vor.

Eine Abschätzung der Rayleighzahl mit Gleichung \ref{eq:Ra1} zeigt, dass Fingerbildung zu erwarten ist:
\begin{equation}
 \mathrm{Ra} = 7341
\end{equation}



\widegraph[./plot/Rayleighnumber]{Abschätzung, ob sich mit dem \COT Experiment mit porösem Medium theoretisch konvektive Prozesse, wie Fingerbildung, beobachten lassen. Als Grenzwert gilt $Ra > 4\pi^2$. Ist die diffusiv gebildete Schicht (h) aus gelöstem \COT groß genug, ist dieses Kriterium erfüllt. Für $D$ wurde $D_{eff}$ nach Gleichung \ref{eq:Deff} gewählt. Die Indizes an den Porenradien beziehen sich auf die Kugelgrößen in Tabelle \ref{tab:kug}}{fig:Ra}

\section{\COTm Experiment mit porösem Medium.}
\label{sec:cpm}

Für diesen Versuch wird ein Aufbau ähnlich wie bei \cite{feustel} verwendet, der aus einem porösen Medium mit räumlich stark heterogenen 
Eigenschaften besteht. Hierbei ist die große \HSC mit Glaskügelchen verschiedener Größen gefüllt. Dabei entstehen Regionen, die sich jeweils über einen Teil der Zellbreite erstrecken, wie in Abbildung \ref{fig:het} zu sehen ist. 
Die Größen der verwendeten Kugeln (\textit{SiLi-Beads}) der Firma \textit{Sigmund-Lindner GmbH} sind in Tabelle \ref{tab:kug} notiert. 

\smallgraph[./plot/hetero]{Die heterogene Struktur des aufgeschütteten porösen Mediums. Die verschiedenen Bereiche unterschiedlicher Kugelgröße lassen sich mit Hilfe von Tabelle \ref{tab:kug} zuordnen. Diese Aufnahme wurde mit einem vorgeschalteten \SI{630}{\nano\meter}-Filter gemacht, wodurch die Ausbreitung besser hervorgehoben wird.}{fig:het}

\begin{table}[b]
  \begin{tabularx}{\linewidth}{X|c|c|c}
		& Durch\-messer 			& Stdabw. d. \O{}			& Raumgewicht	\\
		& $\left[\si{\milli\meter}\right]$	& $\left[\si{\milli\meter}\right]$	& $\left[\si{\kg\per\dm\tothe{3}}\right]$ \\
    \hline\hline
    \circled{1}	& 0,07 - 0,11				& 0,06					& 1,37 \\
    \circled{2}	& 0,2 - 0,3				& 0,03					& 1,44 \\
    \circled{3}	& 0,4 - 0,6				& 0,21					& 1,47 
  \end{tabularx}
  \caption{Daten der verwendeten \textit{SiLi-Beads}. Die Materialdichte der Kugeln betragt \SI[round-precision=2]{2,5}{\kg\per\dm\tothe{3}}. Entnommen aus \cite{feustel}.}
  \label{tab:kug}
\end{table}

% Für dieses Experiment wurde der Zufluss am unteren Rand der Zelle so eingerichtet, dass sowohl \COT als auch Wasser oder gelöstes \BB kontrolliert in die Zelle geleitet werden können. \BB ist eine Nahrungsmittelfarbe, welche hier in einer Konzentration von \SI{0,05}{\gram\per\liter} verwendet wird. Der Farbstoff absorbiert im Wellenlängenbereich von \SI{630}{\nano\meter} maximal. Eine \BB Lösung kann also gut mit dem passenden Filter vor der Kamera verfolgt werden. Siehe auch hierzu Abbildung \ref{fig:het}. Um Lufteinschlüsse zwischen den Glaskugeln beim Befüllen der Zelle zu vermeiden wird vor dem Fluten mit Wasser \COT durch die Zelle gespült. Dazu wird eine Flasche mit etwas Trockeneis befüllt und anschließend das entstehende Gas in die Zelle geleitet. Dazu wird die Flasche, mit der Öffnung leicht nach unten geneigt, auf den Kasten, in dem sich die Zelle befindet gelegt. Ein Schlauch verbindet Flaschenöffnung und untere Zufluss, sodass das Gas einfach durch die Zelle hindurchfließt, da gasförmiges \COT schwerer als Luft ist. Nach etwa ein bis zwei Stunden wird angenommen, dass die gesamte Luft aus der Zelle verdrängt wurde und es wird Wasser zugeführt. In diesem löst sich das Gas, sodass das poröse Medium komplett mit einer Wasser-\COTm Lösung gefüllt ist. Um unerwartete Effekte, ggf. ausgelöst durch das \COT, zu vermeiden wird anschließend noch mehrmals Wasser durch die Zelle gepumpt, damit möglichst kein gelöstes \COT in der Zelle zurückbleibt. Zum Beginn des Experiments wird der Wasserspiegel auf Höhe der obersten Kugelschicht gesenkt und die Pumpschläuche entfernt.
% Die Kamera filmt das Experiment mit einer Bildrate von \SI{1}{Bild\per\hour}. Ein Skript steuert die Beleuchtung, sodass diese nur während der Bildaufnahme angeschaltet ist.

% In einem Versuch die Aufbauten des Verdunstungsexperiments und des \COTm Experiments zu kombinieren, wird die \HSC ähnlich wie in Teil \ref{sec:eva} beschrieben mit den dort verwendeten Glaskügelchen befüllt. 

Die Zelle wird von unten mit der \BCG-Lösung befüllt. Dies geschieht möglichst langsam, um Lufteinschlüsse zu vermeiden.
Zusätzlich zu den in Tebelle \ref{tab:kug} beschriebenen Kügelchen, wurden in einem weiteren Test Borosilikatkugeln mit \BCG zusammengebracht. Das verwendete Borosilikatglas ist chemisch beständiger gegenüber Wasser als die für das Verdunstungsexperiment verwendeten \citep{sili:bor}. Die Eigenschaften der Kugeln sind in Tabelle \ref{tab:bor} festgehalten.

\begin{table}
 \begin{tabularx}{\linewidth}{X|c|c|c}
		& Durch\-messer 			& Stdabw. d. \O{}			& Raumgewicht	\\
		& $\left[\si{\milli\meter}\right]$	& $\left[\si{\milli\meter}\right]$	& $\left[\si{\kg\per\dm\tothe{3}}\right]$ \\
  \hline\hline
  \circled{4}	& 1,0					& 0,3					& 1,10
 \end{tabularx}
 \caption{Daten der verwendeten \textit{SiLi-Beads Typ P}. Die Materialdichte der Kugeln betragt \SI[round-precision=2]{2,23}{\kg\per\dm\tothe{3}}. Entnommen aus \cite{sili:bor}.}
 \label{tab:bor}
\end{table}

Um Abzuschätzen, ob auch bei diesem Experiment Fingerbildung zu beobachten sein sollt wird wieder die Rayleighzahl berechnet, dieses Mal mit Gleichung \ref{eq:Ra2} und in Abhängigkeit der Höhe, der durch die Diffusion entstandenen Schicht aus Wasser mit darin gelöstem \COT. Die Ergebnisse sind in Abbildung \ref{fig:Ra} festgehalten. Man kann erkennen, dass bereits ab relativ dünnen Schichten die kritische Rayleighzah von $4\pi$ überschritten wird. Diese Darstellung lässt allerdings keine zeitliche Einschätzung zu. Der Zeitpunkt, wann $Ra=4\pi$ erreicht wird hängt vom Diffusionskoeffizienten $D$ ab.