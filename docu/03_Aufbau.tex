
\label{sec:set}

Da für beide in Teil \ref{sec:intro} beschriebenen Fragestellungen die Dynamik der betrachteten Systeme interessant ist, wird jeweils ein ``Light Transmission Experiment''
durchgeführt. Hierzu wird eine Hele-Shaw Zelle vor einer homogenen Lichtquelle plaziert. Das Licht, dass die Zelle durchdringt wird anschließend von einer Digitalkamera aufgezeichnet
und für die spätere Auswertung gespeichert.
Größter Unterschied bei den Experimenten ist vor Allem die Dauer. Während das Verdunstungsexperiment fast zwei Wochen dauert ist das \COT-Experiment auf maximal wenige Stunden ausgelegt. 
\TODO{muss der letzte Satz sein?}
\widegraph[./plot/Aufb_Dunst.png]{Grundsätzlicher Aufbau der beiden durchgeführten Experimente. Zu sehen sind: 1: Lichtquelle, 2: Hele-Shaw-Zelle, 3: Kamera \TODO{Beschriftung und passendes Bild}}{fig:auf}

\section{Hele-Shaw Zelle}
\label{sec:hsc}
Grundsätzlich besteht die Hele-Shaw Zelle aus zwei Glasplatten, die einem kleinen Abstand zueinander gehalten werden. Bei diesem Aufbau sind drei der vier Seiten abgedichtet, sodass kein Wasser abfließen kann. 
Die offene Seite der Zelle zeigt nach oben. Auf diese Weise soll dafür gesorgt werden, dass die auftreten Strömungseffekte annähernd zweidimensional sind. 
\TODO{schönere Formulierung}, \TODO{Beleg!}
Die Abmessungen der verwendeten Zellen sind in Tabelle \ref{tab:Hdim} festgehalten.
Am unteren Ende der Zelle befindet sich ein Ausfluss, über den die Zelle kontroliert mit Wasser oder einer gewünschten Lösung befüllt werden kann.


\inkscape{./plot/cell_dimensions.pdf_tex}{\linewidth}{Dimensionierung der Hele-Shaw Zelle. Ansicht von oben und von der Seite. 1:Keil, 2: Dichtung und Abstandshalter, 3: Rahmen, 4: Glasplatte, 5: Füllung der Zelle.}{fig:Hdim}

\begin{table}[h]
  \begin{tabular}{p{2cm}|c|c|c}
    Aufbau			& Höhe				& Breite			& Abstand \\
    \hline\hline
    Verdunstungs\-experiment	& \SI{ 500}{\milli\meter}	& \SI{273}{\milli\meter}	& \SI{3}{\milli\meter} \\
    \COT-Experiment		& \SI{ 250}{\milli\meter}	& \SI{273}{\milli\meter}	& \SI{2,1}{\milli\meter}  
  \end{tabular}
  \caption{Dimensionierung der Hele-Shaw Zellen für die beiden durchgeführten Experimente. Siehe auch Abbildung \ref{fig:Hdim}.}
  \label{tab:Hdim}
\end{table}


\section{Kamera}
\label{sec:cam}
Die Messung wird mit Hilfe einer Kamera durchgeführt. \TODO{Modell, Objektiv}. Der Sensor des benutzten Modells zeichnet sich durch seine stark linearen Eigenschaften aus, wie schon bei \cite{buchner}

\section{Lichtquelle}
\label{sec:light}


\section{Das Verdunstungsexperiment}
\label{set:eva}

Hier kommt eine KURZE Beschreibung des Verdunstungsexperiments hin!

\graph{Aufbau des Verdunstungsexperiments. Zu Sehen sind 1. , 2. , ...}{gr:suevp}
Für das Verdunstungsexperiment wird die Hele-Shaw Zelle mit Glasküglechen verschiedener Größen gefüllt. Dabei wurde darauf geachtet, dass die immer homogene Regionen enstehen. Die Größen der verwendeten Kugeln der Firma Sili-Beads\TODO{Referenz und Tabelle und tm Logo oder so} sind in Tabelle \ref{ref:kug} notiert. Die Porosität
dieser Regionen kann über die Radien der Kügelchen abgeschätzt werden, wie es auch bei \cite{feustel} gemacht wurde. \TODO{original quelle finden und verwenden}
Wie in Teil \ref{theo:por} beschrieben, führen damit die Kugelgrößen zu verschiedenen Leitfähigkeiten. Auch diese sind in Tabelle \ref{tab:kug} festgehalten.
Eine sehr genaue Beschreibung des Aufbaus kann bei \cite{buchner,heberle} nachgelesen werden.


\section{Das \COT-Experiment}
\label{set:cot}
%\subsubsection{\(\COT Experiment in porous media\)}
%\label{set:cpm}
