
\label{cha:res}
\section{Verdunstungsexperiment}
\label{res:eva}


\section{\COT-Experiment}
\label{res:cot}

\graph{Mit Hilfe der diskreten Fourieranalyse wird bestimmt wo sich die Finger befinden. Zusätzlich erhält man die dominierenden Abstände der Finger. Man kann gut erkenne, dass dieser bei \TODO{$k = ??$} liegt. Das bereinigte Spektrum wird zur Fingerdetektion benutzt.}{fig:four}


\graph{Um zu demonstrieren, dass die Methode zur Längenbestimmung funktioniert wurden in diesem Bild die Längen der Finger als Balken über ihre Position geplottet. Da die Finger aber nur in der Anfangsphase des Experiments gerade nach unten sinken kann man in Abbildung \ref{fig:len2} ein Beispiel sehen, wo die Länge der Finger offensichtlich unterschätzt wird.}{fig:len}

\graph{Nach \TODO{$t=\SI{5}{\minute}$} kann die Fingerlänge nicht mehr sinnvoll bestimmt werden.}{fig:len2}

\section{\COT Experiemt im porösen Medium}
\label{res:cpm}
Leider musste schnell festgestellt werden, dass die Kügelchen den pH-Wert des Wassers so sehr basisch beeinflussen, dass das \COT keinen Ausschlag in die saure Richtung verursachen kann. Fotografien von Tests dazu finden sich in Abbildung \ref{fig:CPO}.

\graph{Farbumschläge des \BCG in Verbindung mit verschiedenen Substanzen. \BCG (1) in neutraler Form, \dah im Gleichgewicht mit der umgebenden Luft, (2) in Kombination mit \COT. Man kann sehr gut den Ausschlag ins Gelbe erkennen. (3) mit den Glaskügelchen verschiedener Größen, (4) mit Glaskügelchen und \COT.}{fig:CPO}

Auch die Verwendung von Kügelchen aus anderem Material wurde angedacht. Erste Tests zeigen