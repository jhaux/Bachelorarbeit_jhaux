
\label{cha:con}

In dieser Arbeit wurde gezeigt, wie sich durch das Lösen von \COT in Wasser Dichteinstbilitäten ausbilden, welche zu Fingerbildung und damit \COTm Sequestration führen. Die Beobachtungen wurden ausführlich beschrieben und interpretiert.
Die Experimente wurden in einer \HSC durchgeführt, die für eine zweidimensionale Versuchanordnung sorgt und mit einer Kamera gefilmt.
Mit Hilfe von Python wurden die Bilder ausgewertet und die gewonnen Daten geplottet.
Die Auswertungen lassen erkennen, dass sich das Experiment in drei Phasen gliedert: 
 Diffusion (\SI{0}{\minute} bis \SI{9}{\minute}),
 stabile Fingerbildung (\SI{9}{\minute} bis \SI{60}{\minute}) und
 Vortizitäten auf Zellebene (\SI{60}{\minute} bis zum Ende) 

Während der stabilen Fingerbildung wurde beobachtet, dass die Fingern einen Abstand von \SI[round-precision=2]{1.52}{\centi\meter} zueinander einnehmen und halten, bis Konvektion auf Zellebene für Durchmischung sorgt.

Anhand der Betrachtung eines einzelnen Fingers konnte beobachtet werden, wie die diffusive Schicht mit Beginn der Fingerbildung dünner wird und frisches Wasser an die Oberfläche gelangt. Dieser Prozess begünstigt das Lösungsverhalten von \COT in Wasser, da es so nicht mehr nur von der Diffusivität des \COT im Wasser abhängt. 

Die Detektion der Finger mit der in Teil \ref{sec:ima} beschriebenen Methode weißt eine gewisse Fehleranfälligkeit auf, was ausführlich diskutiert wird. Für eine Verbesserung der Ergebnisse muss sichergestellt werden, dass die Helleigkeitsstabilisierung der aufgenommenen Bilder zuverlässiger funktioniert. 


Ein Versuch das Experiment mit einem porösen Medium durchzuführen ist fehlgeschlagen. Es konnten aber wertvolle Informationen für kommende Versuche dieser Art gewonnen werden. So waren Kugeln aus \KNG ungeeignet für das Experiment, da sie das Wasser basisch machen, wodurch eine Beobachtung der \COTm Bewegung mit Hilfe des \BCGm Indikators unmöglich war. Kugeln aus \BOG hingegen scheinen den Indikator nicht so stark zu beeinflussen.

% \section{Ausblick}

Diese Arbeit gibt nur einen kleinen Einblick in die Thematik der \COTm Sequestration. Um ein besseres Verständnis zu bekommen müssen noch einige Fragen beantwortet werden. 

Um genauer zu verstehen, wann der Übergang von Diffusion zu Konvektion erfolgt lohnt es sich eine höhere zeitliche Auflösung beim Filmen des Experiments zu wählen.

In dieser Arbeit wurde nur das Verhalten bei einer Rayleighzahl betrachtet. Interessant wäre zu beobachten, wie durch Variation der Rayleighzahl das Verhalten der Finger geändert wird. Ein möglicher Ansatz dazu wäre die Spaltbreite zu variieren.

Um wirklich zu verstehen, wie sich \COT unter der Erde in Wasser, welches dort in porösem Gestein vorliegt, verhält lohnt es sich das hier vorgestellte \COTm Experiment mit einem porösen Medium aus \BOG durchzuführen. Von Interesse wäre hier zu sehen, ob und wie sich Finger ausbilden, \dah ob die Beschleunigung des Lösungsvorgangs auch im porösen Medium stattfindet.


Die beschriebenen Methode zur Stabilisierung der Bildhelligkeit muss überarbeitet werden. Offenbar reicht es nicht aus nur einen Messbereich zu wählen und diesen auf der selben Helligkeit, wie die Referenz zu halten. Möglicherweise muss ein zweiter Bereich anderer Helligkeit dazugenommen werden, sodass zur Korrektur auch ein Offset neben dem errechneten Faktor einbezogen werden kann.


Sollten die diskutierten Probleme bei Auswertung und Glaskugeln gelöst werden, hat man mit diesen Versuchsaufbauten, sowohl mit porösem Medium als auch ohne dieses, sehr mächtige und leicht zu handhabende Möglichkeiten \COTm Sequestration unter Laborbedingungen zu untersuchen.