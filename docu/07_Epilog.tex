
\label{cha:con}

In dieser Arbeit wurde gezeigt, wie sich durch das Lösen von \COT in Wasser Dichteinstbilitäten ausbilden, welche zu Fingerbildung und damit \COT-Sequestration führen. Die Beobachtungen wurden ausführlich beschrieben und versucht zu deuten.
Die Experimente wurden in einer \HSC durchgeführt, die für eine prinzipiell zweidimensionale Versuchanordnung sorgt und mit einer Kamera gefilmt.
Mit Hilfe von Python wurden die Bilder ausgewertet und die gewonnen Daten geplottet.
Die Auswertungen lassen erkennen, dass sich das Experiment in drei Phasen gliedert:
\begin{itemize}
 \item Diffusion (\SI{0}{\minute} bis \SI{9}{\minute})
 \item stabile Fingerbildung (\SI{9}{\minute} bis \SI{60}{\minute})
 \item Turbulenz (\SI{60}{\minute} bis zum Ende) 
\end{itemize}


Ein Versuch das Experiment mit einem porösen Medium durchzuführen ist fehlgeschlagen. Es konnten aber wertvolle Informationen für kommende Versuche dieser Art gewonnen werden. So waren Kugeln aus \KNG ungeeignet für das Experiment, da sie das Wasser basisch machen, wodurch eine Beobachtung der \COT Bewegung mit Hilfe des \BCG-Indikators unmöglich war. Kugeln aus \BOG hingegen schienen den Indikator nicht so stark zu beeinflussen.

\section{Ausblick}
\begin{itemize}
 \item Poröses Medium
 \item Methoden verbessern
 \item verschieden dicke \HSCs $\Rightarrow$ verschiedene $Ra$
\end{itemize}
