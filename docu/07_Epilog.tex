
\label{cha:con}

In dieser Arbeit wurde gezeigt, wie sich durch das Lösen von \COT in Wasser Dichteinstabilitäten ausbilden, welche zu Fingerbildung und damit \COTm Sequestration führen. 
Hierzu wurden Experimente in einer \HSC durchgeführt, welche für eine quasi zweidimensionale Versuchsanordnung sorgt. Mittels eines Indikators wurde die Ausbreitung des \COT sichtbar gemacht. Eine Kamera zeichnete den zeitlichen Verlauf des Experiments auf.
% Mit Hilfe von Python wurden die Bilder ausgewertet und die gewonnen Daten geplottet.
Die Auswertung zeigt, dass sich das Experiment in drei Phasen gliedert: 
 (1) Diffusion (\SI{0}{\minute} bis \SI{9}{\minute}),
 (2) stabile Fingerbildung (\SI{9}{\minute} bis \SI{60}{\minute}) und
 (3) Vortizitäten, die die gesamte Zelle umfassen (\SI{60}{\minute} bis zum Ende). 

Im Rahmen dieser Arbeit liegt der Fokus auf der Untersuchung der zweiten Phase, der stabilen Fingerbildung. Während dieser wurde beobachtet, dass die Finger einen Abstand von \SI[round-precision=2]{1.52}{\centi\meter} zueinander einnehmen und solange halten, bis Konvektion auf Zellebene für Durchmischung sorgt.

Anhand der Betrachtung eines einzelnen Fingers konnte beobachtet werden, wie die diffusive Schicht mit Beginn der Fingerbildung dünner wird und frisches Wasser an die Oberfläche gelangt. Dieser Prozess begünstigt das Lösungsverhalten von \COT in Wasser, da das bereits gelöste \COT nicht nur diffusiv, sonder auch konvektiv von der Wasseroberfläche abtransportiert wird, wo sich anschließend neues \COT lösen kann. 

Es wurde eine Methode zur Detektion und Vermessung der Finger entwickelt und ihre Fehler in Kapitel \ref{cha:res} diskutiert. Für eine Verbesserung der Ergebnisse muss sichergestellt werden, dass die Helligkeitsstabilisierung der aufgenommenen Bilder zuverlässiger funktioniert. Offenbar reicht es nicht aus, nur einen Messbereich zu wählen und diesen auf derselben Helligkeit wie die Referenz zu halten. Es wird erwartet, dass ein zweiter Bereich anderer Helligkeit dazugenommen werden muss, sodass zur Korrektur auch ein Offset neben dem errechneten Faktor einbezogen werden kann. Damit ist eine genauere Justierung der Helligkeit sichergestellt.



% Ein Versuch das Experiment mit einem porösen Medium durchzuführen ist fehlgeschlagen. Es konnten aber wertvolle Informationen für kommende Versuche dieser Art gewonnen werden. So waren Kugeln aus \KNG ungeeignet für das Experiment, da sie das Wasser basisch machen, wodurch eine Beobachtung der \COTm Bewegung mit Hilfe des \BCGm Indikators unmöglich war. Kugeln aus \BOG hingegen scheinen den Indikator nicht so stark zu beeinflussen.

% \section{Ausblick}

% Diese Arbeit gibt nur einen kleinen Einblick in die Thematik der \COTm Sequestration. Um ein besseres Verständnis zu bekommen müssen noch einige Fragen beantwortet werden. 
Neben den präsentierten Ergebnissen gibt es noch weiter unbantwortete Fragen, die auch von Interesse sein können:
% Diese Arbeit fügt der Thematik der \COTm Sequestration neue Details hinzu, will aber auch auf einige Punkte hinweisen, die weitere interessante Fragen beantworten könnten:

% Um genauer zu verstehen, wann der Übergang von Diffusion zu Konvektion erfolgt lohnt es sich eine höhere zeitliche Auflösung beim Filmen des Experiments zu wählen.

Das gezeigte Experiment lässt sich zum Beispiel dadurch erweitern, dass durch eine höhere Bildrate beim Aufzeichnen des Experiments eine bessere zeitliche Auflösung des Übergangs von Diffusion zu Konvektion ermöglicht würde. 

% In dieser Arbeit wurde nur das Verhalten bei einer Rayleighzahl betrachtet. Interessant wäre zu beobachten, wie durch Variation der Rayleighzahl das Verhalten der Finger geändert wird. Ein möglicher Ansatz dazu wäre die Spaltbreite zu variieren.

Um ein die Realität besser beschreibendes Bild davon zu bekommen, wie sich in Wasser gelöstes \COTn, welches in porösem Gestein im Boden vorliegt, verhält, wäre es von Interesse, das hier vorgestellte \COTm Experiment mit einem heterogenen porösen Medium aus \BOG durchzuführen. 
Um ein heterogenes Medium realisieren zu können, wurde versucht das Experiment mit einem in der \HSC eingefüllten porösen Medium durchzuführen. Aufgrund technischer Herausforderungen konnte dies im Rahmen dieser Arbeit nicht mehr realisiert werden. Die aufgetretenen Herausforderungen und möglichen Lösungen sind in Kapitel \ref{res:cpm} aufgezeigt.
Von Interesse wäre hier, zu vergleichen, wie sich Finger ausbilden, \dah wie sich die Beschleunigung des Lösungsvorgangs durch Heterogenitäten im porösen Medium ändert.

Zusammen mit den vorgeschlagenen Lösungen hat man mit diesen Versuchsaufbauten, sowohl mit heterogenem porösen Medium als auch ohne dieses, sehr mächtige und leicht zu handhabende Möglichkeiten, \COTm Sequestration unter Laborbedingungen zu untersuchen.