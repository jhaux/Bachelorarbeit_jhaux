\documentclass[oneside, a4paper, DIV=11]{scrartcl}

% PACKAGES
\usepackage[ngerman]{babel}

%SETTINGS


%TITLE
\title{Density induced transport in a Hele-Shaw Experiment}
\author{Johannes Haux}
\date{1.4.2015}

%DOCUMENT
\begin{document}
\maketitle

\section{Introduction}
% Gutes Verständis, da Bodenphysik immer gute Beisiele hat:
% Salzsee, CO2 einlagerung

Während der Durchführung meiner Bachelorarbeit entstanden zunächst Ideen zu einem Experiment, dass sich schließlich als nicht durchführbar herausstellte, für den zur 
Verfügung stehenden Zeitraum. In Folge dessen kam die Idee zu einem weiteren, für die zur Verfügung stehende Zeit besser geeignetem Experiment. Aus diesem Grund teilt
sich diese Arbeit in jedem ihrer Abschnitte immer in inhaltlich dem einen, wie dem anderen Experiment zugehörigen Bereiche.

Grundlegend für alle Fragestellungen, die im Laufe dieser Arbeit aufkamen, sind Dichteinstabilitäten, die zur treibenden Kraft von Prozessen werden, die mit Hilfe
einer Hele-Shaw-Zelle beobachtet werden sollen.
Zunächst wird die Frage gestellt, wie sich Verdunstungsphänomene auf Stofftransport in gesättigten, heterogenen, porösen Medien auswirken und zu Stofftransport 
von der Oberfläche in tiefere Schichten führt. Als Beispiel kann man sich einen Salzsee vorstellen, der dabei ist auszutrocknen und dabei Salz in tiefere Erdschichten
einlagert.
In einem zweiten Ansatz wird die Frage gestellt, wie sich in Wasser lösendes \COT für Dichteinstabilitäten sorgt, die schließlich das gelöste Gas in tiefer Wasserbereiche 
führt. Auch hier lässt sich wieder ein sehr Anwendungsbezogenes Beispiel finden, wie schon \cite{fernandez} treffend festgestellt hat: Das Einlagern von \COT in
Gesteinsschichten setzt vorraus, dass sich das \COT lange genug auf dem Gestein aufhält. Sorgt man dafür, dass unterirdische Wasserreservoirs mit \COT gesättigt werden
kann man dieses Verhalten künstlich herbeiführen. Ein Verständnis dafür, wie sich \COT in Wasser löst und bewegt ist dafür grundlegend.

Diese Arbeit ist gegliedert in folgende Teile: Zunächst soll in Teil \ref{sec:theo} eine theoretische Grundlage geschaffen werden, zum Verständis der folgenden Abschnitte.
Anschließend erkläre die Teile \ref{sec:set} "Experimenteller Aufbau" und \ref{sec:ima} "Bildanalyse" die Methoden, mit denen Messdaten beschaffen und ausgewerted wurden.
Die Ergebisse dieser Messungen werden in Teil \ref{sec:res} "Ergebnisse" präsentiert und diskutiert.
Am Ende folgt eine Zusammenfassunge mit Ausblick.


\section{Grundlagen}
\label{sec:theo}
\subsection{Poröse Medien}
\label{theo:por}

\TODO{porosität + leitfähigkeit}
\subsection{Fingerbildung}
\label{theo:fing}
\subsection{}


\section{Experimenteller Aufbau}
\label{sec:set}

Grundlegender Baustein der beiden durchgeführten Experimente ist die Hele-Shaw Zelle. Unterschied bei den Experimenten ist vor Allem die Dauer. Während das
Verdunstungsexperiment fast zwei Wochen dauert ist das \COT-Experiment auf maximal wenige Stunden ausgelegt. Im folgenden werden die jeweiligen Aufbauten beschrieben.
\TODO{muss der letzte Satz sein?}

\subsection{Die Hele-Shaw Zelle}
\label{sec:hsc}
Grundsätzlich besteht die Hele-Shaw Zelle aus zwei 
Glasplatten, die einem kleinen Abstand zueinander gehalten werden. Bei diesem Aufbau sind drei der vier Seiten abgedichtet, sodass kein Wasser abfließen kann.
Die offene Seite der Zelle zeigt nach oben. Auf diese Weise soll dafür gesorgt werden, dass die auftreten Strömungseffekte annähernd zweidimensional sind.
\TODO{schönere Formulierung}, \TODO{Beleg!}


\subsection{Das Verdunstungsexperiment}
\label{set:eva}

Hier kommt eine KURZE Beschreibung des Verdunstungsexperiments hin!

\graph{}{Aufbau des Verdunstungsexperiments. Zu Sehen sind 1. , 2. , ...}
Für das Verdunstungsexperiment wird die Hele-Shaw Zelle mit Glasküglechen verschiedener Größen gefüllt. Dabei wurde darauf geachtet, dass die immer homogene Regionen
enstehen. Die Größen der verwendeten Kugeln der Firma Sili-Beads\TODO{Referenz und Tabelle und tm Logo oder so} sind in Tabelle \ref{ref:kug} notiert. Die Porosität
dieser Regionen kann über die Radien der Kügelchen abgeschätzt werden, wie es auch bei \cite{feustel} gemacht wurde. \TODO{original quelle finden und verwenden}
Wie in Teil \ref{sec:por} beschrieben, führen damit die Kugelgrößen zu verschiedenen Leitfähigkeiten. Auch diese sind in Tabelle \ref{tab:kug} festgehalten.


\subsection{Das \COT-Experiment}
\label{set:cot}
%\subsubsection{\(\COT Experiment in porous media\)}
%\label{set:cpm}

\section{Image Analysis}
\label{sec:ima}


\section{Results}
\label{sec:res}
\subsection{Das Verdunstungsexperiment}
\label{res:eva}
%\subsection{\COT Experiemt im porösen Medium}
%\label{res:cpm}
\subsection{\COT-Experiment}
\label{res:cot}

\section{Zusammenfassung - schönermachen!}
\label{sec:con}
\subsection{Verdunstungsexperiment}
\label{con:eva}
\subsection{\COT-Experiment}
\label{con:cot}

\section{Literature}
\label{sec:lit}

\end{document}
