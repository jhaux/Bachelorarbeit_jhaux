\documentclass[oneside, a4paper, DIV=11]{scrartcl}

% PACKAGES
\usepackage[ngerman]{babel}

%SETTINGS


%TITLE
\title{Density induced transport in a Hele-Shaw Experiment}
\author{Johannes Haux}
\date{1.4.2015}

%DOCUMENT
\begin{document}
\maketitle

\section{Introduction}
% Gutes Verständis, da Bodenphysik immer gute Beisiele hat:
% Salzsee, CO2 einlagerung

Diese Arbeit ist gegliedert in folgende Teile: Zunächst soll in Teil \ref{sec:theo} eine theoretische Grundlage geschaffen werden, 



\section{Theory}
\label{sec:theo}
\subsection{porous media}
\label{theo:por}
\subsection{"fingering"}
\label{theo:fing}
\subsection{}


\section{Experimental Setup}s
\label{sec:set}
\subsection{Evaporation Experiment}
\label{set:eva}
\subsection{\COT Experiment}
\label{set:cot}
\subsubsection{\(\COT Experiment in porous media\)}
\label{set:cpm}

\section{Image Analysis}
\label{sec:ima}


\section{Results}
\label{sec:res}
\subsection{Evaporation Experiment}
\label{res:eva}
\subsection{\COT Experiemtn in porous media}
\label{res:cpm}
\subsection{\COT Experiment}
\label{res:cot}

\section{Conclusion}
\label{sec:con}
\subsection{Evaporation Experiment}
\label{con:eva}
\subsection{\COT Experiment}
\label{con:cot}

\section{Literature}
\label{sec:lit}

\end{document}
