% - Motivation
%   - ich will Bachelorabreit schreiben
%   - Ansatz 1: Verständnis von Prozesse
% - ein bisschen Theorie
%   - Porosität
%   - Fingering und Rayleigh-Zahl
% - was bisher geschah
%   - Verdunstung
%   - CO2 ohne Kugeln
%   - CO2 mit Kugeln


%--------------------------------------------------------------------------
% Titlepage
%--------------------------------------------------------------------------
\maketitle
%\begin{frame}[plain]
%	\titlepage
%\end{frame}

%--------------------------------------------------------------------------
% Intro
%--------------------------------------------------------------------------
\section*{Introduction}

\begin{frame}
	\frametitle{Motivation}
	\vspace{1cm} % generate some space between title and content
	\begin{itemize}
	\pause
	  \item Update on my work
	\pause
	  \item Feedback for me
	\end{itemize}	
\end{frame}
\note[itemize]{
       \item Ich will Euch auf den neuesten Stand meiner Arbeit bringen
       \item Dazu erklären, was ich schon gemacht habe $\Rightarrow$ chronologischer Aufbau
       \item Vorallem will ich feedback: \textbf{Was sind interessante Aspekte, die ich mir noch ansehen kann?}
}

%--------------------------------------------------------------------------
% Table of contents
%--------------------------------------------------------------------------
\section*{Outline}
\begin{frame}{Outline}
	% hideallsubsections ist empfehlenswert für längere Präsentationen
	\tableofcontents[hideallsubsections]
\end{frame}


%--------------------------------------------------------------------------
% Evaporation
%--------------------------------------------------------------------------
\section{Evaporation}

\subsection{General Idea}
\begin{frame}
	\frametitle{General idea}
% 	\vspace{1cm} % generate some space between title and content
	
	\begin{columns}[t]
		\column{5cm}
		\vspace{1cm}
		\begin{itemize}
		  \item heterogeneous, porus medium
		\end{itemize}
		
		\column{5cm}
		\begin{figure}
		  \centering
		  \includegraphicscopyright[width=2.5cm]{Aufb_Dunst_00}{Selfmade graphic}
		  \caption{Heterogeneous, porous medium}
		\end{figure}
      \end{columns}
\end{frame}
\note[itemize]{
       \item Schritt 1: heterogenes poröses Medium. Hier Glaskugeln. Analogie: Erdschicht
       \item Schritt 2: Füllen mit Wasser. \textbf{Achtung!} Keine Lufteinschlüsse $\Rightarrow$ CO2
       \item Schritt 3: Warten... Verdunstung dauert.
}

\begin{frame}
	\frametitle{General idea}
% 	\vspace{1cm} % generate some space between title and content
	
	\begin{columns}[t]
		\column{5cm}
		\vspace{1cm}
		\begin{itemize}
		  \item heterogeneous, porus medium
		  \item saturated with water
		\end{itemize}
    
		\column{5cm}
		\begin{figure}
		  \centering
		  \includegraphicscopyright[width=2.5cm]{Aufb_Dunst_01}{Selfmade graphic}
		  \caption{Heterogeneous, porous medium with water}
		\end{figure}
      \end{columns}
\end{frame}
\note[itemize]{
       \item Schritt 1: heterogenes poröses Medium. Hier Glaskugeln. Analogie: Erdschicht
       \item Schritt 2: Füllen mit Wasser. \textbf{Achtung!} Keine Lufteinschlüsse $\Rightarrow$ CO2
       \item Schritt 3: Warten... Verdunstung dauert.
}

\begin{frame}
	\frametitle{General idea}
% 	\vspace{1cm} % generate some space between title and content
	
	\begin{columns}[t]
		\column{5cm}
		\vspace{1cm}
		\begin{itemize}
		  \item heterogeneous, porus medium
		  \item saturated with water
		  \item evaporation
		\end{itemize}
		
		\column{5cm}
		\begin{figure}
		  \centering
		  \includegraphicscopyright[width=2.5cm]{Aufb_Dunst_02}{Selfmade graphic}
		  \caption{Porous medium with water and evaporation}
		\end{figure}
      \end{columns}
\end{frame}
\note[itemize]{
       \item Schritt 1: heterogenes poröses Medium. Hier Glaskugeln. Analogie: Erdschicht
       \item Schritt 2: Füllen mit Wasser. \textbf{Achtung!} Keine Lufteinschlüsse $\Rightarrow$ CO2
       \item Schritt 3: Warten... Verdunstung dauert.
}

\begin{frame}
	\frametitle{General idea}
% 	\vspace{1cm} % generate some space between title and content
	
	\begin{columns}[t]
		\column{5cm}
		\vspace{1cm}
		\begin{itemize}
		  \item heterogeneous, porus medium
		  \item saturated with water
		  \item evaporation
		  \item water is pumped up
		  \end{itemize}
		
		\column{5cm}
		\begin{figure}
		  \centering
		  \includegraphicscopyright[width=2.5cm]{Aufb_Dunst_03}{Selfmade graphic}
		  \caption{Porous medium with water and evaporation}
		\end{figure}
      \end{columns}
\end{frame}
\note[itemize]{
       \item Schritt 1: heterogenes poröses Medium. Hier Glaskugeln. Analogie: Erdschicht
       \item Schritt 2: Füllen mit Wasser. \textbf{Achtung!} Keine Lufteinschlüsse $\Rightarrow$ CO2
       \item Schritt 3: Warten... Verdunstung dauert.
       \item Schritt 4: Was wollen wir beobachten: Wasser wird hochgepumpt. Darüber Abschätzung der Zeitskala
}

\subsection{Setup}
\begin{frame}
	\frametitle{Setup}
% 	\vspace{1cm} % generate some space between title and content
	
	\begin{figure}
	  \centering
	  \includegraphicscopyright[width=\textwidth]{Aufb_Dunst}{Selfmade graphic}
	  \vspace{-0.75cm}
	  \caption{Experimental setup}
	\end{figure}
 
\end{frame}

\subsection{Image Analysis}
\begin{frame}
	\frametitle{Image Analysis}
% 	\vspace{1cm} % generate some space between title and content
	
	\begin{figure}
	  \centering
	  \includegraphicscopyright[width=\textwidth]{image_analysis}{Selfmade graphic}
	  \vspace{-0.75cm}
	  \caption{Image analysis}
	\end{figure}
 
\end{frame}
\note[itemize]{
       \item erklären was man sieht
       \item warum refine? BSP: kameraintensität schwankt -> Sensor warm?
       \item Man kann Pulsieren beobachten: Zufluss leicht verstopft?
       \item Resultat: Über einen längeren Zeitraum (2 Wochen) kann man Verdunstungsprozesse gut beobachten.
       \item Aber: Für diese Bachelorarbeit ein sehr langer Prozess.
}


\subsection{Video}
\begin{frame}
	\frametitle{Video}
	\vspace{1cm} % generate some space between title and content
	
	\centering
	Let's have a look.
 
\end{frame}
\note[itemize]{
       \item Zelle schon stark in Mittleidenschaft gezogen...
       \item Es wird ein Puls BB hineingegeben: Wasser - BB - Wasser
       \item Man kann Pulsieren beobachten: Zufluss leicht verstopft?
       \item Resultat: Über einen längeren Zeitraum (2 Wochen) kann man Verdunstungsprozesse gut beobachten.
       \item Aber: Für diese Bachelorarbeit ein sehr langer Prozess.
}




%--------------------------------------------------------------------------
% CO2
%--------------------------------------------------------------------------

\section{\COT}
\subsection{General Idea}

\begin{frame}
	\frametitle{General idea}
% 	\vspace{1cm} % generate some space between title and content
	
	\begin{columns}[t]
		\column{5cm}
		\vspace{1cm}
		\begin{itemize}
		  \item heterogeneous, porus medium
		\end{itemize}
		
		\column{5cm}
		\begin{figure}
		  \centering
		  \includegraphicscopyright[width=2.5cm]{Aufb_CO2_00}{Selfmade graphic}
		  \caption{Heterogeneous, porous medium}
		\end{figure}
      \end{columns}
\end{frame}

\begin{frame}
	\frametitle{General idea}
% 	\vspace{1cm} % generate some space between title and content
	
	\begin{columns}[t]
		\column{5cm}
		\vspace{1cm}
		\begin{itemize}
		  \item heterogeneous, porus medium
		  \item saturated with water
		\end{itemize}
    
		\column{5cm}
		\begin{figure}
		  \centering
		  \includegraphicscopyright[width=2.5cm]{Aufb_CO2_01}{Selfmade graphic}
		  \caption{Heterogeneous, porous medium with water}
		\end{figure}
      \end{columns}
\end{frame}

\begin{frame}
	\frametitle{General idea}
% 	\vspace{1cm} % generate some space between title and content
	
	\begin{columns}[t]
		\column{5cm}
		\vspace{1cm}
		\begin{itemize}
		  \item heterogeneous, porus medium
		  \item saturated with water
		  \item \COT on top
		\end{itemize}
		
		\column{5cm}
		\begin{figure}
		  \centering
		  \includegraphicscopyright[width=2.5cm]{Aufb_CO2_02}{Selfmade graphic}
		  \caption{Porous medium with water and CO2}
		\end{figure}
      \end{columns}
\end{frame}

\begin{frame}
	\frametitle{General idea}
% 	\vspace{1cm} % generate some space between title and content
	
	\begin{columns}[t]
		\column{5cm}
		\vspace{1cm}
		\begin{itemize}
		  \item heterogeneous, porus medium
		  \item saturated with water
		  \item \COT on top
		  \item Fingering
		  \end{itemize}
		
		\column{5cm}
		\begin{figure}
		  \centering
		  \includegraphicscopyright[width=2.5cm]{Aufb_CO2_03}{Selfmade graphic}
		  \caption{Porous medium with water and CO2}
		\end{figure}
      \end{columns}
\end{frame}
\note[itemize]{
       \item Schritt 1: heterogenes poröses Medium. Hier Glaskugeln. Analogie: Erdschicht
       \item Schritt 2: Füllen mit Wasser. \textbf{Achtung!} Keine Lufteinschlüsse $\Rightarrow$ SEHR langsam
       \item Schritt 3: Warten... Verdunstung dauert.
       \item Schritt 4: Was wollen wir beobachten: Fingering durch Dichte instabilitäten
}

\subsection{Setup}
\begin{frame}
	\frametitle{Setup}
% 	\vspace{1cm} % generate some space between title and content
	
	\begin{figure}
	  \centering
	  \includegraphicscopyright[width=\textwidth]{Aufb_CO2}{Selfmade graphic}
	  \vspace{-0.75cm}
	  \caption{Experimental setup}
	\end{figure}
 
\end{frame}
\note[itemize]{
       \item Aufbau wie vorher, \textbf{ABER:} \COT kommt oben rein!
       \item \COT schwerer als Luft $\Rightarrow$ Flasche mit Trockeneis reicht
       \item Schritt 3: Warten... Verdunstung dauert.
       \item Schritt 4: Was wollen wir beobachten: Wasser wird hochgepumpt. Darüber Abschätzung der Zeitskala
}

\subsection{Indicator}
\begin{frame}
	\frametitle{Indicator}
% 	\vspace{1cm} % generate some space between title and content
	
	To see where the dissolved \COT is, a pH-Indicator can be used.
	
	\pause
	
	\COT in equilibrium with water:
	\begin{equation*}
	  \mathrm{CO_2 + H_2O\ \rightleftharpoons \ H_2CO_3}
	\end{equation*}
	
	\pause
	\centering
	$\Downarrow$
	\begin{align*}
	  \mathrm{pH}(\mathrm{water} \rightleftharpoons \mathrm{air}) &\approx 5.6 \\
	  \mathrm{pH}(\mathrm{water} \rightleftharpoons \mathrm{CO_2}) &\approx 3.9
	\end{align*}


\end{frame}
\note[itemize]{
       \item Warum ist das so?
       \item \COT löst sich im Wasser und lässt Kohlensäure entstehen
       \item Änderung des pH werts
       \item Also: passernder Indikator gesucht!
}

\begin{frame}
	\frametitle{Bromocresol Green}
% 	\vspace{1cm} % generate some space between title and content
	
	\begin{figure}
	  \centering
	  \includegraphicscopyright[width=8cm]{Bromocresol_green_spectrum}{Graph from wikipedia \cite{Spec:BCG}}
	  \caption{Bromocresol Green spectrum}
	\end{figure}
\end{frame}
\note[itemize]{
       \item BCG gut geieignet, da passend zu Filtern an der Kamera
       \item Umschlag klar sichtbar!
}

\subsection{Expectations}
\begin{frame}
	\frametitle{Expectations}
	\vspace{1cm} % generate some space between title and content
	
	\centering
	Dissolved \COT is heavier than pure water
	
	\pause
	$\Downarrow$
	
	Unstable layering
	
	\pause
	$\Downarrow$
	
	Diffusive/Convectiv process
	
\end{frame}

\begin{frame}
	\frametitle{Expectations}
	\vspace{1cm} % generate some space between title and content
	
	Handle: Rayleigh number Ra as in Kneafsy et al \cite{2010:CO2}:
	
	\begin{equation*}
	  Ra = \frac{Kgh \Delta\rho}{D}
	\end{equation*}

	
\end{frame}
\note[itemize]{
       \item $K=\frac{r_0^2}{8\mu}$, $\frac{m^3s}{kg}$
       \item $\frac{D_{eff}^{diff}}{D_m}=\frac{\theta}{\Phi^{\frac{2}{3}}}=\Phi^{\frac{1}{3}}$, da $\frac{\theta}{\Phi} = 1$
       \item $\Delta \rho = \left| \rho_{H_2O} - \rho_{CO_{2, aq}} \right| = 10 \frac{kg}{m^3}$
}

\begin{frame}
	\frametitle{Expectations}
	\vspace{1cm} % generate some space between title and content
	
	Conductivity is dependend on the size of the glass beads and thus also Ra(h,r):
	
	\begin{figure}
	  \centering
	  \includegraphicscopyright[width=\textwidth]{Rayleighnumber}{Selfmade graphic}
	  \caption{Experimental setup}
	\end{figure}

	
\end{frame}
\note[itemize]{
       \item Kneafsy et al: crit = $4\pi^2$
       \item effektiv zieht in betracht, dass die Diffusion im Porösen Medium schlechter ist (effektive Weglänge)
       \item sollte gut gehn!
}




\subsection{Video}
\begin{frame}
	\frametitle{Video}
	\vspace{1cm} % generate some space between title and content
	
	\centering
	Let's have a look.
 
\end{frame}
\note[itemize]{
       \item Test super!
       \item Experiment blöd!
}


\subsection{Debugging}
\begin{frame}
	\frametitle{Debugging}
	\vspace{1cm} % generate some space between title and content
	\begin{figure}
	  \centering
	  \begin{minipage}[t]{0.45\textwidth}
	    \includegraphicscopyright[width=\textwidth]{ph_dirty}{graphic selfmade}
	  \end{minipage}
	  \begin{minipage}[t]{0.45\textwidth}
	    \includegraphicscopyright[width=\textwidth]{ph_clean}{graphic selfmade}
	  \end{minipage}
	  \caption{Different pH-values for different glas beads}
	\end{figure}
 
\end{frame}
\note[itemize]{
       \item Waschen scheint was zu bringen?
}


\begin{frame}
	\frametitle{Debugging}
	\vspace{1cm} % generate some space between title and content
	\begin{figure}
	  \centering
	  \begin{minipage}[t]{0.3\textwidth}
	    \includegraphicscopyright[width=\textwidth]{BCG_test_00}{graphic selfmade}
	  \end{minipage}
	  \begin{minipage}[t]{0.3\textwidth}
	    \includegraphicscopyright[width=\textwidth]{BCG_test_01}{graphic selfmade}
	  \end{minipage}
	  \begin{minipage}[t]{0.3\textwidth}
	    \includegraphicscopyright[width=\textwidth]{BCG_test_02}{graphic selfmade}
	  \end{minipage}
	  
	  \caption{Glas beads in Bromocresol Green}
	\end{figure}
 
\end{frame}
\note[itemize]{
       \item Kügelchen = mega basisch!
}


\begin{frame}
	\frametitle{Debugging}
% 	\vspace{1cm} % generate some space between title and content
	\begin{figure}
	  \centering
	  \includegraphicscopyright[width=0.75\textwidth]{BCG_test_03}{graphic selfmade}
	  \caption{Glas beads in Bromocresol Green overview}
	\end{figure}
 
\end{frame}
\note[itemize]{
       \item Experiment so nicht durchführbar!
}

\section{Future}
\begin{frame}
	\frametitle{Future}
	\vspace{1cm} % generate some space between title and content
	\begin{figure}
	  \centering
	  \doublequoted{blackboard}
	\end{figure}

\end{frame}

%\section{Danke}



\section{Literature}
\begin{frame}{Literaturverzeichnis}
	\begin{thebibliography}{10}
    
	\beamertemplateonlinebibitems
	\bibitem{Spec:BCG}
	Wikipedia
	\newblock \doublequoted{Bromocresol green}
	\newblock http://en.wikipedia.org/wiki/Bromocresol\_green

	\beamertemplatearticlebibitems
	\bibitem{2010:CO2}
	Springerlink.com
	\newblock \doublequoted{Laboratory Flow Experiments for Visualizing Carbon Dioxide-Induced, Density-Driven Brine Convection}
	\newblock 2010
  \end{thebibliography}
\end{frame}

