%!TEX program = xelatex

% Name           : hsrm-beamer-demo.sty
% Author         : Benjamin Weiss (benjamin.weiss@kreatiefton.de)
% Version        : 0.4
% Created on     : 05.05.2013
% Last Edited on : 24.03.2014
% Copyright      : Copyright (c) 2013-2014 by Benjamin Weiss. All rights reserved.
% License        : This file may be distributed and/or modified under the
%                  GNU Public License.
% Description    : HSRM beamer theme demonstration. Also includes a short 
%                  Tutorial regarding the beamer class.

\documentclass[compress]{beamer}
%--------------------------------------------------------------------------
% Common packages
%--------------------------------------------------------------------------
\usepackage[english]{babel}

\usepackage{graphicx}
\graphicspath{ {./pres/pics/}{./plot/pic/} }
\usepackage{multicol}
% Erweiterte Tabellenfunktionen
\usepackage{tabularx,ragged2e}
\usepackage{booktabs}
% Listingserweiterung
\usepackage{listings}
\lstset{ %
language=[LaTeX]TeX,
basicstyle=\normalsize\ttfamily,
keywordstyle=,
numbers=left,
numberstyle=\tiny\ttfamily,
stepnumber=1,
showspaces=false,
showstringspaces=false,
showtabs=false,
breaklines=true,
frame=tb,
framerule=0.5pt,
tabsize=4,
framexleftmargin=0.5em,
framexrightmargin=0.5em,
xleftmargin=0.5em,
xrightmargin=0.5em
}
\usepackage{subscript} % Tiefegestellter Text außerhalb des Mathemodus
\usepackage{amsmath}

%--------------------------------------------------------------------------
% Load theme
%--------------------------------------------------------------------------
\usetheme{hsrm}

% \usepackage{dtklogos} % must be loaded after theme
% \usepackage{tikz}
% \usetikzlibrary{mindmap,backgrounds}

%--------------------------------------------------------------------------
% General presentation settings
%--------------------------------------------------------------------------
\title{Density driven flow in porous media}
\subtitle{status and goals}
\date{16.02.2015}
\author{Johannes Haux}
\institute{Institute of Environmental Physics\\University of {\Medium Heidelberg}}

%--------------------------------------------------------------------------
% Notes settings
%--------------------------------------------------------------------------
\setbeameroption{hide notes}


\begin{document}
% Custom makros like \COT for CO2
\ProvidesPackage{mymacros}[2015/03/14 v1.0 My own macros]

\newcommand{\graph}[3][./src/missing_graph.eps]{
  \begin{figure}[H]
      \centering
      \includegraphics[width=\linewidth]{#1}
      \caption{#2}
      \label{#3}
  \end{figure}
}

\newcommand{\TODO}[1]{
  \textbf{\textcolor{red}{#1}}
}

\newcommand{\COT}{CO\textsubscript{2} }

% All the content
% - Motivation
%   - ich will Bachelorabreit schreiben
%   - Ansatz 1: Verständnis von Prozesse
% - ein bisschen Theorie
%   - Porosität
%   - Fingering und Rayleigh-Zahl
% - was bisher geschah
%   - Verdunstung
%   - CO2 ohne Kugeln
%   - CO2 mit Kugeln


%--------------------------------------------------------------------------
% Titlepage
%--------------------------------------------------------------------------

%\begin{frame}[plain]
%	\titlepage
%\end{frame}

%--------------------------------------------------------------------------
% Intro
%--------------------------------------------------------------------------
\section{Introduction}

\begin{frame}
	\frametitle{Motivation}
	\vspace{1cm} % generate some space between title and content
	\begin{itemize}
	\pause
	  \item Update on my work
	\pause
	  \item Feedback for me
	\end{itemize}	
\end{frame}
\note[itemize]{
       \item Ich will Euch auf den neuesten Stand meiner Arbeit bringen
       \item Dazu erklären, was ich schon gemacht habe $\Rightarrow$ chronologischer Aufbau
       \item Vorallem will ich feedback: \textbf{Was sind interessante Aspekte, die ich mir noch ansehen kann?}
}

%--------------------------------------------------------------------------
% Table of contents
%--------------------------------------------------------------------------
\section*{Contents}
\begin{frame}{Gliederung}
	% hideallsubsections ist empfehlenswert für längere Präsentationen
	\tableofcontents[hideallsubsections]
\end{frame}


%--------------------------------------------------------------------------
% First idea
%--------------------------------------------------------------------------
\section{Evaporation}

\begin{frame}
	\frametitle{General idea}
	\vspace{1cm} % generate some space between title and content
	
	Setup: Wet (saturated) soil with a groundwater reservoir.
	\begin{columns}[t]
		\column{3cm}
		\begin{itemize}
     
		\item heterogeneous, porus medium
     
    
		\end{itemize}
		\column{7cm}
		\begin{figure}
		\centering
		\includegraphicscopyright[width=3cm]{Aufb_Dunst_00}{Graphik selbst erstellt}
		\caption{Heterogeneous, porous medium}
		\end{figure}
      \end{columns}
\end{frame}
\note[itemize]{
       \item Schritt 1: heterogenes poröses Medium. Hier Glaskugeln. Analogie: Erdschicht
       \item Schritt 2: Füllen mit Wasser. \textbf{Achtung!} Keine Lufteinschlüsse $\Rightarrow$ CO2
       \item Schritt 3: Warten... Verdunstung dauert.
}

\begin{frame}
	\frametitle{General idea}
	\vspace{1cm} % generate some space between title and content
	
	Setup: Wet (saturated) soil with a groundwater reservoir.
	\begin{columns}[t]
		\column{3cm}
		\begin{itemize}
     
		\item heterogeneous, porus medium
		\item saturated with water
     
    
		\end{itemize}
    
		\pause
    
		\column{7cm}
		\begin{figure}
		\centering
		\includegraphicscopyright[width=3cm]{Aufb_Dunst_01}{Graphik selbst erstellt}
		\caption{Heterogeneous, porous medium with water}
		
		\end{figure}
      \end{columns}
\end{frame}
\note[itemize]{
       \item Schritt 1: heterogenes poröses Medium. Hier Glaskugeln. Analogie: Erdschicht
       \item Schritt 2: Füllen mit Wasser. \textbf{Achtung!} Keine Lufteinschlüsse $\Rightarrow$ CO2
       \item Schritt 3: Warten... Verdunstung dauert.
}

\begin{frame}
	\frametitle{General idea}
	\vspace{1cm} % generate some space between title and content
	
	Setup: Wet (saturated) soil with a groundwater reservoir.
	\begin{columns}[t]
		\column{3cm}
		\begin{itemize}
     
		\item heterogeneous, porus medium
		\item saturated with water
		\item evaporation
     
    
		\end{itemize}
    
		\pause
    
		\column{7cm}
		\begin{figure}
		\centering
		\includegraphicscopyright[width=3cm]{Aufb_Dunst_02}{Graphik selbst erstellt}
		\caption{Heterogeneous, porous medium with water and evaporation}
		
		\end{figure}
      \end{columns}
\end{frame}
\note[itemize]{
       \item Schritt 1: heterogenes poröses Medium. Hier Glaskugeln. Analogie: Erdschicht
       \item Schritt 2: Füllen mit Wasser. \textbf{Achtung!} Keine Lufteinschlüsse $\Rightarrow$ CO2
       \item Schritt 3: Warten... Verdunstung dauert.
}

\begin{frame}
	\frametitle{General idea}
	\vspace{1cm} % generate some space between title and content
	
	Setup: Wet (saturated) soil with a groundwater reservoir.
	\begin{columns}[t]
		\column{3cm}
		\begin{itemize}
     
		\item heterogeneous, porus medium
		\item saturated with water
		\item evaporation
		\item water is pumped up
     
    
		\end{itemize}
    
		\pause
    
		\column{7cm}
		\begin{figure}
		\centering
		\includegraphicscopyright[width=3cm]{Aufb_Dunst_02}{Graphik selbst erstellt}
		\caption{Heterogeneous, porous medium with water and evaporation}
		
		\end{figure}
      \end{columns}
\end{frame}
\note[itemize]{
       \item Schritt 1: heterogenes poröses Medium. Hier Glaskugeln. Analogie: Erdschicht
       \item Schritt 2: Füllen mit Wasser. \textbf{Achtung!} Keine Lufteinschlüsse $\Rightarrow$ CO2
       \item Schritt 3: Warten... Verdunstung dauert.
       \item Schritt 4: Was wollen wir beobachten: Wasser wird hochgepumpt. Darüber Abschätzung der Zeitskala
}

\begin{frame}
	\frametitle{Setup}
	\vspace{1cm} % generate some space between title and content
	
	Hier kommt das blender Bild hin
	Dünner Schlauch von unten!
 
\end{frame}

\begin{frame}
	\frametitle{Image Analysis}
	\vspace{1cm} % generate some space between title and content
	
	
 
\end{frame}

\begin{frame}
	\frametitle{Video}
	\vspace{1cm} % generate some space between title and content
	
	Let's have a look.
 
\end{frame}
\note[itemize]{
       \item Zelle schon stark in Mittleidenschaft gezogen...
       \item Es wird ein Puls BB hineingegeben: Wasser - BB - Wasser
       \item Man kann Pulsieren beobachten: Zufluss leicht verstopft?
       \item Resultat: Über einen längeren Zeitraum (2 Wochen) kann man Verdunstungsprozesse gut beobachten.
       \item Aber: Für diese Bachelorarbeit ein sehr langer Prozess.
}


\section{CO$_2$}
\begin{frame}
	\frametitle{General idea}
	\vspace{1cm} % generate some space between title and content
	
	Setup: Wet (saturated) soil with a groundwater reservoir.
	\begin{columns}[t]
		\column{3cm}
		\begin{itemize}
     
		\item heterogeneous, porus medium
		\item saturated with water
		\item CO$_2$ Layer on top
		\item Fingering
     
    
		\end{itemize}
    
		\pause
    
		\column{7cm}
		\begin{figure}
		\centering
		\includegraphicscopyright[width=3cm]{Aufb_Dunst_01}{Graphik selbst erstellt}
		\caption{Heterogeneous, porous medium with water}
		
		\end{figure}
      \end{columns}
\end{frame}
\note[itemize]{
       \item Zelle schon stark in Mittleidenschaft gezogen...
       \item Es wird ein Puls BB hineingegeben: Wasser - BB - Wasser
       \item Man kann Pulsieren beobachten: Zufluss leicht verstopft?
       \item Resultat: Über einen längeren Zeitraum (2 Wochen) kann man Verdunstungsprozesse gut beobachten.
       \item Aber: Für diese Bachelorarbeit ein sehr langer Prozess.
}



\section{Literature}
\begin{frame}{Literaturverzeichnis}
	\begin{thebibliography}{10}
    
	\beamertemplatebookbibitems
	\bibitem{Oppenheim2009}
	Alan~V.~Oppenheim
	\newblock \doublequoted{Discrete-Time Signal Processing}
	\newblock Prentice Hall Press, 2009

	\beamertemplatearticlebibitems
	\bibitem{EBU2011}
	European~Broadcasting~Union
	\newblock \doublequoted{Specification of the Broadcast Wave Format (BWF)}
	\newblock 2011
  \end{thebibliography}
\end{frame}

\section{Ausblick}
\begin{frame}{Bekannte Fehler}
	\begin{itemize}
		\item Theme ist momentan noch in einer einzelnen sty-Datei. Diese sollte unterteilt werden in einzelne Dateien für Schrift, Farbe usw.
	\end{itemize}
\end{frame}

\begin{frame}{Fragen, Anmerkungen, Kontakt}
	Das HSRM Theme steht unter der \quoted{GNU Public License}. Es darf also weitergegeben und modifiziert werden, sofern die Lizenzart beibehalten wird.
	
	Für Fragen und Anmerkungen stehe ich gerne zur Verfügung.
	\begin{itemize}
		\item \url{Benjamin.Weiss@student.hs-rm.de}
	\end{itemize}
\end{frame}

\end{document}






