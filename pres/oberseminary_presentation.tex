%!TEX program = xelatex

% Name           : hsrm-beamer-demo.sty
% Author         : Benjamin Weiss (benjamin.weiss@kreatiefton.de)
% Version        : 0.4
% Created on     : 05.05.2013
% Last Edited on : 24.03.2014
% Copyright      : Copyright (c) 2013-2014 by Benjamin Weiss. All rights reserved.
% License        : This file may be distributed and/or modified under the
%                  GNU Public License.
% Description    : HSRM beamer theme demonstration. Also includes a short 
%                  Tutorial regarding the beamer class.

\documentclass[compress]{beamer}
%--------------------------------------------------------------------------
% Common packages
%--------------------------------------------------------------------------
\usepackage[german]{babel}

\usepackage{graphicx}
\graphicspath{{./pics}{../plot/pic}}
\usepackage{multicol}
% \usepackage[utf8]{inputenc}
% \usepackage{enumitem} %Anpassbare Enumerates/Itemizes
% Erweiterte Tabellenfunktionen
\usepackage{tabularx,ragged2e, amsmath, siunitx}
\usepackage{booktabs}
% Listingserweiterung
\usepackage{listings}
\lstset{ %
language=[LaTeX]TeX,
basicstyle=\normalsize\ttfamily,
keywordstyle=,
numbers=left,
numberstyle=\tiny\ttfamily,
stepnumber=1,
showspaces=false,
showstringspaces=false,
showtabs=false,
breaklines=true,
frame=tb,
framerule=0.5pt,
tabsize=4,
framexleftmargin=0.5em,
framexrightmargin=0.5em,
xleftmargin=0.5em,
xrightmargin=0.5em
}

\usepackage{caption}
% \captionsetup[figure]{slc=off}
%--------------------------------------------------------------------------
% Load theme
%--------------------------------------------------------------------------
\usetheme{hsrm}

\usepackage{dtklogos} % must be loaded after theme
\usepackage{tikz}
\usetikzlibrary{mindmap,backgrounds}

%--------------------------------------------------------------------------
% General presentation settings
%--------------------------------------------------------------------------
\title{Der Zeeman Effekt}
\subtitle{Ein FP-Versuch}
\date{27. Januar 2015}
\author{Johannes Haux}
\institute{FP Physik\\Universität {\Medium Heidelberg}}

%--------------------------------------------------------------------------
% Notes settings
%--------------------------------------------------------------------------
\setbeameroption{hide notes}
% \setbeameroption{show only notes}

\begin{document}
%--------------------------------------------------------------------------
% Titlepage
%--------------------------------------------------------------------------

\maketitle

%\begin{frame}[plain]
%	\titlepage
%\end{frame}

%--------------------------------------------------------------------------
% Table of contents
%--------------------------------------------------------------------------

% - Motivation
%   - ich will Bachelorabreit schreiben
%   - Ansatz: Verständnis von Dichteinduzierten Prozessen in Heterogenen Medien
%   - Beispiel: Verdunstung
%   - Beispiel: Fingering
% - ein bisschen Theorie
%   - Porosität
%   - Fingering und Rayleigh-Zahl
% - was bisher geschah
%   - Verdunstung
%   - CO2 ohne Kugeln
%   - CO2 mit Kugeln
% - Was jetzt?
%   - Neue Kugeln?
%   - Komplett neuer Ansatz
%     - Drop on water -> Wie dringt der Finger ein
%     - ungesättigte Zone -> wie dringt die Wasserfront in den Boden vor? Findet Austausch mit dem bereits vorhandenen Wasser statt?


\section*{Outline}      % * macht die grauen Header Seiten weg
\begin{frame}{Outline}
	% hideallsubsections ist empfehlenswert für längere Präsentationen
	\tableofcontents[hideallsubsections]
\end{frame}

%--------------------------------------------------------------------------
% Content
%--------------------------------------------------------------------------

%-------------------------------------------------------------------
%                          Motivation
%-------------------------------------------------------------------
%
\note[itemize]{\item empty}

\section{Motivation}
\begin{frame}
    \frametitle{Motivation}

    \vspace{1cm} % generate some space between title and content
     
    \pause
    
    \begin{columns}[t]
    
    \column{3cm}
    \begin{itemize}
     
     \item experimentelle Bachelorabreit
     
    
    \end{itemize}
    
    \pause
    
    \column{7cm}
    \begin{figure}
     \centering
     \includegraphicscopyright[width=3cm]{exite}{Graphik entnommen aus dem Internet \cite{INet:exite}}
     \includegraphicscopyright[width=4cm]{Spektrum2}{Graphik entnommen aus dem Internet \cite{INEt:Spec}}
     \caption{Spektroskopie}
     
    \end{figure}
    \vspace{1em}

    \end{columns}
    
    

 
\end{frame}
\note[itemize]{
       \item Bachelorabreit mit Experiment: Basteln macht Spaß!
}

\begin{frame}
    \frametitle{Motivation}

    \vspace{1cm} % generate some space between title and content
\end{frame}


%-------------------------------------------------------------------
%                          Section 8
%-------------------------------------------------------------------
%

\section*{Literatur}
% \begin{frame}[allowframebreaks]
%     \frametitle{Literatur}
% 
%     \vspace{1cm} % generate some space between title and content
%     \bibliographystyle{plain}
%     \bibliography{Literatur}
%  
% \end{frame}

\begin{frame}
    \frametitle{Literatur}

    \vspace{0.2cm} % generate some space between title and content
    \begin{thebibliography}{10}
	
	\beamertemplatearticlebibitems
        \bibitem{tungstenNIST}
        Journal of Research of the National Bureau of Standards. Section A: Physics and Chemistry
	\newblock \doublequoted{The First Spectrum of Tungsten (W I)}
	\newblock 1968
	
	\beamertemplatearticlebibitems
        \bibitem{F44Skript}
        Anleitungen zum Fortgeschrittenenpraktikum der Universt\"at Heidelberg
	\newblock \doublequoted{Zeeman Effekt}
	\newblock 2012
	
	\beamertemplatearticlebibitems
        \bibitem{dataNIST}
        National Institute of Standard and Technology
	\newblock \doublequoted{NIST Database}
	\newblock 2014
	
	\beamertemplatearticlebibitems
        \bibitem{valueNIST}
        National Institute of Standard and Technology
	\newblock \doublequoted{NIST Reference on Constant, Units and Uncertainty}
	\newblock 2014
	
	


    \end{thebibliography}

 
\end{frame}

\begin{frame}
    \frametitle{Literatur}

    \vspace{0.2cm} % generate some space between title and content
    \begin{thebibliography}{10}
	
	\beamertemplateonlinebibitems
        \bibitem{Wiki:Uhr}
        Wikipedia
	\newblock \doublequoted{Atomuhr}
	\newblock {\small http://de.wikipedia.org/wiki/Atomuhr}
	
	\beamertemplateonlinebibitems
        \bibitem{Apod:ISS}
        Astronomy Picture of the Day
	\newblock \doublequoted{Interior View}
	\newblock {\small http://apod.nasa.gov/apod/ap150123.html}
	
	\beamertemplateonlinebibitems
        \bibitem{INet:Katze}
        einfachtierisch.de
	\newblock \doublequoted{So können Sie Ihre Katze erziehen}
	\newblock {\small http://www.einfachtierisch.de/katzen/katzenerziehung/so-koennen-sie-ihre-katze-erziehen-id33485/}
	
	\beamertemplateonlinebibitems
        \bibitem{INet:exite}
        Mortimer Abramowitz, Matthew J. Parry-Hill, Robert T. Sutter, and Michael W. Davidson
	\newblock \doublequoted{Electron Excitation and Emission}
	\newblock {\small http://www.olympusmicro.com/primer/java/fluorescence/exciteemit/}
	
	


    \end{thebibliography}

 
\end{frame}

\begin{frame}
    \frametitle{Literatur}

    \vspace{0.2cm} % generate some space between title and content
    \begin{thebibliography}{10}
	
	\beamertemplateonlinebibitems
        \bibitem{INet:Spec}
        lehrer-online
	\newblock \doublequoted{Spektrum einer Energiesparlampe}
	\newblock {\small http://www.lehrer-online.de/834196.php}

	
	


    \end{thebibliography}

 
\end{frame}

%-------------------------------------------------------------------
%                          Section 9
%-------------------------------------------------------------------
%

\section{Anhang}
\begin{frame}
    \frametitle{Lorentzfit vs Gaussfit}

    \vspace{1cm} % generate some space between title and content
    Lorentzprofil:
    \begin{itemize}
     \item Entspricht der natürlichen Linienbreite, verursacht durch die Lebensdauer des angeregten Zustandes: $\Delta E \Delta t \gtrapprox \frac{\hbar}{2}$
    \end{itemize}
    Gaussprofil:
    \begin{itemize}
     \item Durch thermische Bewegung wird die Linie aufgeweitet, da ein Dopplereffekt entsteht. Hängt von der Temperatur ab.
    \end{itemize}
 
 
\end{frame}





\end{document}






